\documentclass{report}
\usepackage{xcolor}
\usepackage{hyperref}
\usepackage{pdfpages} 
\usepackage{changepage}
\usepackage{parskip}
\usepackage{graphicx}
\usepackage{setspace}
\doublespacing
\usepackage{rotating}
\usepackage{tabto}
\usepackage{geometry}
\usepackage{enumitem}
\usepackage{scrextend}
\usepackage{xepersian}
\setlatintextfont{Times New Roman}
\settextfont[Scale=1]{XB Niloofar}	
%\setlength{\parindent}{4em}

\begin{document}
\renewcommand\partname{فاز}
\includepdf[pages=-]{start}

\tableofcontents
\newpage

\part{تحلیل و طراحی معماری}
\chapter{استخراج نیازمندی‌های نرم‌افزار}
\section{مقدمه}
\subsection{هدف}
هدف از شرح نیازمندی‌های نرم‌افزار و تدوین سند
\lr{SRS}\LTRfootnote{Software Requirements Specification}
ارائه‌ی دید جامع درباره‌ی پروژه به مخاطب است؛ به این معنی که بتوان با مطالعه این سند کلیتی راجع به نیازمندی‌های کارکردی، غیر کارکردی، قیود طراحی و سایر مشخصات سامانه به‌دست آورد.
مخاطبین این سند به‌طور کلی اشخاص زیر هستند:
\renewcommand\labelitemi{\tiny$\bullet$}
\begin{itemize}
  \item   دانشگاه‌ها 
  \item   مدیران و معاونین دانشکده‌ها 
  \item اساتید
  \item دانشجویان
\end{itemize}
\subsection{قلمرو}
نام محصول درحال توسعه «کندو» می‌باشد.کندو جهت آسان‌سازی عملیات مرتبط با تنظیم برنامه‌ی درسی و انتخاب واحد و به‌‌طورکلی در جهت رفاه حال دانشجویان و معاونین آموزشی دانشگاه‌ها طراحی شده است. این سامانه به دانشجویان برای تنظیم یک برنامه‌ی درسی بهینه در طول ترم، کارشناسان آموزش دانشگاه برای مشاهده و پیگیری دقیق امور انتخاب واحد از طریق ارائه‌ی گزارش های متنوع و نهایتاً اساتید برای برنامه‌ریزی بهتر در ترم پیش‌رو کمک می کند. سامانه کندو در وهله اول برای سیستم دانشگاهی دانشگاه اصفهان در نظر گرفته شده است ولی قابلیت توسعه برای تمامی دانشگاه‌های ایران را دارد.
\subsection{ تعاریف، سرنام‌ها، و کوته‌نوشت‌ها }
  در این قسمت برخی از سرنام‌ها و تعریف مفاهیم مهم و پرتکرار برای آشنایی بیشتر با آن‌ها  آورده شده است:
  \begin{itemize}
  \item
وب‌سرور
\LTRfootnote{WebServer}
\\
   میزبان یک نرم‌افزار، یا سخت‌افزاری برای اجرای یک نرم‌افزار است که امکان پاسخگویی به درخواست‌های کاربران شبکه جهانی وب را دارد.

\item
زبان 
\lr{html}
\\
 نام این زبان برگرفته از حروف اول عبارت 
\lr{Hyper Text Markup Language}
  می‌باشد و به عنوان یک زبان نشانه‌گذاری متن (زبان استاندارد صفحات وب) استفاده می‌شود.

\item
زبان 
\lr{css}
\\
یک زبان برنامه‌نویسی ظاهری برای صفحات وب است که برای ارائه زیباتر و قابل فهم‌تر یک صفحه وب نوشته شده و به عنوان یک زبان نشانه‌گذاری به کار برده می‌شود.


\item
سند
\lr{SRS}
\\
سند شرح نیازمندی‌های نرم افزار که به اختصار سند
\lr{SRS} 
 نیز نامیده می‌شود، سندی است که چالش‌ها، ویژگی‌ها، و آنچه که نیاز است تا به واسطه نرم‌افزار برطرف شود را توصیف می‌کند.

\item
\lr{Microsoft SQL Server}
\\
یک سیستم مدیریت بانک‌های اطلاعاتی رابطه‌ای است که توسط شرکت مایکروسافت ساخته شده‌ است. این سیستم به عنوان یک سرور پایگاه داده، یک محصول نرم‌افزاری است که عملکرد اصلی‌اش ذخیره‌سازی و بازیابی اطلاعات می‌باشد.





\item
فایروال
 \LTRfootnote{FireWall}
 \\
فایروال یا دیوارآتش نام عمومی برنامه‌هایی است که از دستیابی غیرمجاز به یک سیستم رایانه جلوگیری می‌کنند. در برخی از این نرم‌افزارها، برنامه‌ها بدون اخذ مجوز قادر نخواهند بود از یک رایانه برای سایر رایانه‌ها، داده ارسال کنند.


  \end{itemize}
\subsection{مراجع}
   مهندسی نرم‌افزار شئ‌گرا یک متدولوژی چابک یکنواخت/تألیف دیوید کونگ؛ ترجمه بهمن زمانی و افسانه فاطمی؛ انتشارات دانشگاه اصفهان، 1395.
\subsection{طرح کلی}
در ادامه این سند به بررسی ویژگی‌ها، قابلیت‌ها و محدودیت‌های سامانه‌ی کندو پرداخته می‌شود. هم‌چنین، توضیحاتی پیرامون این‌که این سامانه چگونه با سامانه‌های دیگر دانشگاهی همکاری دارد ارائه خواهد شد. شرحی نیز بر نحوه‌ی کمک کندو به دانشجویان و معاونین آموزشی دانشگاه که منجر به تسهیل فعالیت و افزایش کارایی آن‌ها
می‌شود داده خواهد شد.
  \section{شرح کلی}
  سامانه کندو برای کمک به دانشجویان و معاونین آموزشی دانشکده‌ها واساتید برای رفع مشکلات موجود در زمان انتخاب واحد طراحی شده است. این سامانه به دانشجویان کمک می‌کند که انتخاب واحد مؤثرتر و آسان‌تری را تجربه کنند، برنامه درسی خود را به شکل بهتر و با سهولت بیشتر تنظیم کنند و هم‌چنین گزارش‌های مفیدی را پیرامون  برنامه‌ی انتخابی دانشجویان در اختیار معاونین آموزشی هر دانشکده  قرار می‌دهد تا آن‌ها نیز بتوانند دروس هر ترم را با دید وسیع‌تر و مطابق با خواسته و نیاز  دانشجویان ارائه دهند.
  \subsection{چشم‌انداز محصول}
     هدف اصلی این سیستم ایجاد سهولت و کارایی بیشتر در فرآیند انتخاب واحد، تنظیم برنامه درسی دانشجویان و هم‌چنین کمک به معاونین  آموزشی دانشکده‌ها با هدف ارائه‌ی بهتر دروس می‌باشد تا مشکلات مربوط به انتخاب واحد و تعیین دروس ارائه‌شده برای هر ترم تا حد زیادی کاهش یابد و حتی برطرف شود.
 \subsubsection{واسط‌های سیستم}
 سامانه‌ی کندو برای انجام فرآیندهای داخلی خود اعم از چیدمان برنامه‌های هفتگی و یا ارائه‌ی گزارش‌های آن‌ها به معاونین آموزشی، نیازمند اطلاعات گوناگونی همچون مشخصات و وضعیت تحصیلی دانشجویان، مشخصات دروس و چارت‌های آموزشی دانشکده‌ها، ویژگی‌های اساتید و نتایج ارزشیابی آن‌ها و... می‌باشد، که عمده‌ی این اطلاعات را از سیستم آموزشی گلستان و سیستم ثانیه دریافت می‌کند و یا به شکل دستی توسط کاربران سیستم وارد می‌شود. درنهایت این داده‌ها را به شکل کارآمد و مجزایی در بانک‌های اطلاعاتی خود که پایگاه‌های  داده‌
 \LTRfootnote{DataBase} 
 نامیده می‌شوند ذخیره می‌کند (در بخش
\hyperlink{one}{\underline{\lr{R9}}}
از
نیازمندی‌های کارکردی
 به این بانک‌ها اشاره شده است) . هم‌چنین پس از انجام وظایف و خدمات خود نتایج حاصل را که همان خروجی‌های سامانه هستند، در اختیار کاربران و سیستم‌های دیگر اعم از سیستم آموزشی گلستان و سامانه‌ی مدیریت یادگیری 
\LTRfootnote{LMS}
  قرار می‌دهد.
 
     \subsubsection{واسط‌های کاربر}
     کندو یک نرم‌افزار تحت‌وب می‌باشد که در آن تعامل کاربران با سیستم از طر یق وب‌سایت کندو صورت می‌گیرد. اولویت‌های تیم توسعه در طراحی واسط کاربری, با توجه به این که مخاطبان اصلی کندو قشر جوان و علمی  جامعه می‌باشند به ترتیب کارایی هر چه بیشتر, نوآوری در طراحی,  زیبایی و درنهایت سادگی در استفاده بوده است. در بخش
\hyperlink{ee}{\underline{واسط‌های نرم‌افزاری}}
     به اسامی ابزارهای  بکار گرفته شده برای طرا حی واسط کاربری اشاره شده است
 و در بخش
\hyperlink{ww}{\underline{\lr{R10}}}
 از نیازمندی‌های کارکردی به انواع کاربران سیستم اشاره شده است که واسط کاربری برای هر کدام از آن‌ها دارای تفاوت‌هایی جزیی می‌باشد. 
 
\subsubsection{واسط‌های سخت‌افزاری}
برای راه‌اندازی کندو از یک سرور مرکزی استفاده می‌شود که برای افزایش کارایی سرور، بر روی آن دو ماشین مجازی پیاده‌سازی می‌شود که این کار منجر به تقسیم منابع سرور بین این دو ماشین مجازی می‌شود. ماشین مجازی اول برای نگهداری بانک‌های اطلاعاتی کندو به کار می‌رود و ماشین  مجازی دوم که درصد بیشتری از منابع سرور را به خود اختصاص می‌دهد، مربوط به وب‌سرور می‌باشد و بر روی آن قرار می‌گیرد. هم‌چنین از یک سرور ثانویه در مکانی متفاوت با مکان سرور مرکزی برای پشتیبان‌گیری 
\LTRfootnote{Backup}
از سرور اصلی استفاده می‌شود که بر روی آن نیز دو ماشین مجازی قرار دارد؛ یکی از آن‌ها برای پشتیبان‌گیری از پایگاه داده و دیگری برای پشتیبان‌گیری از وب‌سرور اصلی پیاده‌سازی می‌شود. (هدف از قرار دادن سرور ثانویه در مکانی خارج از مکان سرور اصلی در نظر گرفتن حوادث احتمالی مثل آتش سوزی توسط تیم توسعه می‌باشد.)
\subsubsection{واسط‌های نرم‌افزاری}
\hypertarget{ee}{سامانه‌ی کندو}
 بر بستر وب پیاده‌سازی می‌شود. در سمت کاربر
\LTRfootnote{Front-End}
    از
 \lr{html, css} و زبان برنامه‌نویسی
\lr{javascript} استفاده می‌گردد؛ برای توسعه‌ی بهتر و سریع‌تر نیز چارچوب‌های
\LTRfootnote{Frameworks}
    \lr{VueJS}
     و 
    \lr{BootStrap}
      به کار می‌روند. 
    در سمت سرور،  اولاً به دلیل اینکه سرور سامانه‌ی ما یک سرور ویندوزی است، وب‌سرور پیاده‌سازی شده روی آن نیز وب‌سرور شرکت ماکروسافت یعنی 
    \lr{ISS}
     است. هم‌چنین برای پیاده‌سازی وب‌سایت و ارتباط آن با سرور
 از 
\lr{asp.net}
 استفاده می‌شود، دوم اینکه برای پایگاه‌ داده‌های کندو
\lr{SQL server}
   شرکت ماکروسافت به کار برده می‌شود.
   دلایل استفاده از این موارد در بخش
\hyperlink{gg}{\underline{قیود طراحی}}
     آمده است.
\subsubsection{واسط‌های ارتباطی}
همان‌طور که گفته شد کندو یک نرم‌افزار تحت وب است که بر روی یک وب‌سرور قرار دارد. نحوه‌ی ارتباط اجزای مختلف کندو با یکدیگر به‌وسیله‌ی کوئری زدن به وب سرور و درخواست و دریافت اطلاعات می‌باشد. برای رسیدن به این هدف کندو نیاز به یک شبکه‌ی محلی برای ارتباطات داخلی خود دارد. همچنین کاربران نیز به کمک واسط کاربری پیاده‌سازی شده در وب‌سایت کندو، درخواست‌های خود را برای وب‌سرور ارسال و با آن ارتباط برقرار می‌کنند. درنهایت برای ارتباط کندو با دیگر سیستم‌ها، لازم است که کندو به شبکه‌ی جهانی اینترنت متصل شود و ارتباط بین کندو و دیگر سیستم‌ها از طریق کوئری زدن طرفین شکل بگیرد.
\subsubsection{واسط‌های حافظه}
همان‌طور که در بخش صفات اضافی اشاره شده است، سامانه کندو نیازمند فضایی برای ذخیره‌سازی اطلاعات می‌باشد. داده‌های مختلفی که به‌صورت خودکار و یا به شکل دستی توسط سامانه‌ی کندو دریافت می‌شوند باید در بانک‌های اطلاعاتی (پایگاه‌های داده) ذخیره شوند؛ این داده‌ها برای پردازش فرآیندها مورداستفاده قرار می‌گیرند. با توجه به این‌که پایگاه داده اصلی و ثانویه‌ی کندو هر دو بر روی سرورهای مجازی در حال اجرا می‌باشند، امکان افزایش فضای ذخیره‌سازی به‌راحتی برای آنها وجود دارد و تنها محدودیتی که می‌تواند ایجاد شود‌، محدودیت مالی پروژه می‌باشد؛ این موضوع تعیین خواهد کرد که تیم توسعه برای افزایش فضای ذخیره‌سازی سرورها تا چه حد می‌توانند پیش بروند. (در رابطه با این‌که چند نوع بانک اطلاعاتی در سامانه وجود دارد به‌تفصیل در بخش 
\hyperlink{ww}{\underline{\lr{R9}}}
 از نیازمندی‌های کارکردی اشاره شده است) .
\subsection{کارکرد محصول}
سامانه‌ی کندو به طور کلی وظیفه‌ی تسهیل فرآیند انتخاب واحد برای دانشجویان و معاونین آموزشی را بر عهده دارد. این سامانه درزمینه‌ی تحلیل و بررسی دروسی که باید ارائه شوند به معاونین آموزشی کمک می‌کند و هم‌چنین، با تنظیم برنامه هفتگی دانشجویان انتخاب واحد مطمئن‌تری را برای آنان به ارمغان می‌آورد. ازجمله کارکردهای دیگر سامانه کندو می‌توان به ثبت‌نام مقدماتی، تنظیم برنامه پیشنهادی برای هر دانشجو، امکان چت کردن و ارسال درخواست توسط دانشجویان، نمایش برنامه‌های هفتگی و برنامه‌ی نهایی برای هر دانشجو و درنهایت پایگاه داده‌ای غنی از وضعیت تحصیلی دانشجویان و اطلاعات دروس و اساتید هر رشته اشاره کرد.
\subsection{مشخصات کاربر}
کاربران سامانه ی تنظیم برنامه درسی و مهارت‌های مورد انتظار آن‌ها برای کسب تجربه‌ی مطلوب با کندو به شرح زیر میباشد:
\begin{itemize}
\item دانشجویان\\ 
دانشجویان مقطع کارشناسی و کارشناسی ارشد دانشگاه اصفهان یکی از اصلی ترین ذی‌نفعان و کاربران این سامانه هستند. انتظار میرود دانشجویان، آشنایی مختصری با فرآیند کلی انتخاب واحد و تجربه‌ی کار با سیستم های مشابه را داشته باشند.
\item معاونین آموزشی
\\
مسئولان و کادر آموزشی دانشگاه باید سابقه و تجربه‌ی کافی در زمینه‌ی انتخاب و برنامه‌ریزی دروس ترم پیش‌رو داشته باشند تا بتوانند با تحلیل درست گزارش‌ها، زمان‌بندی‌ها و برنامه‌ی ارائه‌ی دروس را بهبود بخشند. همچنین، انتظار میرود که اطلاعاتی بسیار جامع و کامل پیرامون مشکلات و سوالات مربوط به امور تحصیلی دانشجویان داشته باشند تا پرسش‌وپاسخ‌های انجام‌شده (که برای سایر دانشجویان نیز قابل مشاهده است) بیشترین کارایی را داشته باشد.
\item اساتید
\\
هر استاد برای تکمیل بانک اطلاعاتی مربوط به خود، باید دید جامع و نسبتاً دقیقی از برنامه‌ی تدریس، نحوه‌ی ارزیابی و بارم‌بندی خود درترم پیش‌رو داشته باشد. به‌علاوه، لازم است اساتید راهنما نیز دید جامع و کاملی پیرامون رشته‌ی تحصیلی و چارت پیشنهادی دانشکده‌ی خود داشته باشند تا بتوانند به بهترین شکل پاسخگوی نیازهای آموزشی دانشجویان باشند.
\item
 ادمین‌های سیستم
 \\
این افراد باید اطلاعات کاملی از نحوه‌ی پیاده‌سازی و عیب‌یابی سامانه داشته باشند، به‌علاوه باید قادر باشند به کمک ابزار نظارت
\LTRfootnote{monitoring}
 بر شبکه از رخدادهای ناگهانی که ممکن است در وضعیت شبکه و ارتباطات میان بخش‌های مختلف ایجاد شود، جلوگیری کنند. درنهایت لازم است با قابلیت‌های سیستم و نحوه‌ی کار با واسط کاربری سامانه نیز آشنایی کامل داشته باشند تا بتوانند تغییرات لازم در سایت را به‌درستی اعمال کنند.
\item توسعه‌دهندگان نرم‌افزاری 
 \\
ازآنجاکه سامانه‌ی کندو به‌صورت بهینه، کارآمد و نوآورانه طراحی شده است و همچنین یک سامانه‌ی متن‌باز است، می‌تواند به توسعه‌دهندگان دیگر برای گسترش این نرم‌افزار و یا ساخت نرم‌افزارهای مشابه ایده‌های بسیار خوبی ببخشد. لازم به ذکر است که این افراد باید قادر به مشاهده‌ی بخش‌های مختلف سامانه نیز باشند لذا جز کاربران مهمان سامانه‌ی کندو خواهند بود.
میزان دسترسی و فعالیت‌هایی که هرکدام از کاربران بالا در سامانه‌ی کندو می‌توانند انجام دهند و توضیحات مربوط به حساب‌های کاربری گوناگون در بخش
\hyperlink{ww}{\underline{\lr{R10}}} 
 از نیازمندی‌های کارکردی آورده شده است.
\end{itemize}
\subsection{قیود}
\begin{itemize}
\item\lr{ -C1}
محدودیت مالی
\\
\hypertarget{tt}{هزینه‌ی}
پیاده‌سازی این پروژه نباید از یک میلیارد ریال تجاوز کند، همچنین تقسیم بندی بودجه‌ی پروژه نیز بر عهده‌ی تیم توسعه قرار داده شده است.
\item 
\lr{ -C2}محدودیت زمانی
\\
سامانه باید حداکثر پس از گذشت 9 ماه از زمان عقد قرارداد در اختیار مشتری قرار بگیرد.
\item \lr{ -C3}محدودیت پشتیبانی
\\
با توجه به این‌که توسعه‌ی نرم‌افزار یک مسئله‌ی بدرفتار می‌باشد، تیم توسعه باید به‌صورت مادام‌العمر پشتیبانی پروژه را به عهده بگیرد و درصورتی‌که ادمین های سیستم از مشکلی در سیستم آگاه شدند، تیم توسعه موظف است با فرستادن نماینده‌ی خود نزد مشتری، مشکل به وجود آمده را برطرف سازد.
\end{itemize}
\subsection{مفروضات و وابستگی‌ها}
\begin{itemize}
\item 
برای کار با سامانه و نمایش اطلاعات اتصال به اینترنت نیاز است. 
\item
  کاربران باید توانایی کار با سیستم کامپیوتری و مرورگرهای اینترنت رایج را داشته باشند.
\item
بانک‌های اطلاعاتی سامانه باید دائماً همگام‌سازی
\LTRfootnote{synchronize}
 و به‌روزرسانی شود و کوچکترین تغییرات در سامانه‌ی کندو و دیگر سیستم‌ها باید به سرعت در پایگاه داده‌ی کندو نیز لحاظ شود.
\item
پایگاه داده‌ی دانشجویان نباید هیچ‌گونه اطلاعات نادرستی مانند نمرات، وضعیت مشروطی، لیست دروس پاس شده‌ی دانشجو و... را شامل شود.
\item
اساتید موظف هستند حساب های کاربری خود را به‌روز نگه دارند.
\item
دانشجویان برای تکمیل نظرسنجی اساتید باید مشارکتی قابل‌قبول و غیرمغرضانه داشته باشند تا بانک اطلاعاتی اساتید قابل اطمینان باشد.
\item
معاونین آموزشی و اساتید موظف هستند پاسخ درخواست های دانشجویان را به طور کامل و در اسرع وقت ارسال نمایند.
\item
داده‌های دریافتی توسط سامانه نباید هیچ‌گونه هم‌پوشانی و تناقضی با یک‌دیگر داشته باشند. برای مثال لیست دروس ترمی ارائه‌شده از سامانه‌ی ثانیه نباید دارای تداخل زمانی باشد، چراکه در این صورت پردازش بر روی‌داده‌ها توسط سامانه‌ی کندو به‌درستی صورت نخواهد گرفت.
\end{itemize}
\section{نیازمندی‌های خاص}
استخراج نیازمندی‌ها فرایندی است که طی آن در مورد توانمندی‌های سیستم در حال ساخت تصمیم‌گیری می‌شود. برای این کار باید اطلاعات موجود تحلیل شوند و سپس با مشورت اعضا و تصمیم‌گیری نهایی، نیازمندی‌های لازم استخراج شوند. نیازمندی‌های سامانه‌ی کندو نیز طبق گام‌های مطرح‌شده در کتاب، در ادامه بیان شده است:
\begin{itemize}
\item
گام اول - جمع‌آوری اطلاعات در مورد دامنه و کاربرد :\\
تیم توسعه علاوه بر این‌که از قبل با معضلات سیستم آموزشی موجود آشنا بودند، از طریق ارتباط با افرادی که به شکلی مستقیم یا غیرمستقیم تجربه‌ی کار با این‌گونه سیستم‌ها را داشتند (این ارتباط به کمک طرح پرسش، مصاحبه و دریافت داستان‌های آنها صورت گرفت) و هم‌چنین، مطالعه‌ی ضوابط و شرایط حاکم و درنهایت تحلیل تمامی اطلاعات به دست آمده، تعدادی از مشکلات و کمبودهای سیستم را شناسایی کردند و بعضی از نیازهای سیستم که باید برطرف شوند مشخص شدند.
\item
گام دوم - استخراج نیازمندی‌ها و محدودیت‌ها :\\
پس از بازخوردهایی که از سوی کاربران صورت گرفت و نتایجی که از جمع‌آوری اطلاعات به دست آمد؛ با مشورت اعضای گروه و طبق تصمیم گیری‌های انجام‌شده در این فاز، تعداد زیادی نیازمندی‌های کارکردی گردآوری شد. هم‌چنین، تعدادی از نیازمندی‌های غیر کارکردی که عموماً در پروژه‌ها پنهان می‌مانند در نظر گرفته شد تا مشکلاتی که فراتر از قابلیت‌های سیستم در پردازش اطلاعات هستند نیز لحاظ گردند.
\item
گام سوم - مطالعه‌ی امکان‌سنجی :\\
در بخش مطالعه‌ی امکان‌سنجی پس از بررسی‌های صورت گرفته, تعدادی از نیازمندی‌هایی که در بخش قبل بررسی شده بودند حذف گردید؛ از دلایل حذف این‌گونه نیازمندی‌ها می‌توان به مشکلات و محدودیت‌هایی که در بخش‌های
\hyperlink{tt}{\underline{2.4}} 
 و
 \hyperlink{gg}{\underline{3.3}}
 آمده است، اشاره کرد. با توجه به مطالعه‌ی امکان‌سنجی صورت گرفته در این مرحله، نیازمندی‌هایی که قابلیت برنامه‌ریزی و پیاده‌سازی داشتند باقی ماندند. بااین‌حال، توجه به این نکته ضروری است متدولوژی چابک پیاده‌سازی شده در این پروژه, اضافه شدن نیازمندیهای جدید و تغییر نیازمندی‌های کنونی را میسر می‌سازد.
\item
گام چهارم - مرور شرح نیازمندی‌ها :\\
درنهایت با مرور دوباره‌ای که توسط اعضای تیم توسعه صورت گرفت؛ دامنه، کاربرد،  نیازمندی‌ها و محدودیت‌های به‌دست آمده بار دیگر با شرایط موجود بررسی شدند و درنهایت اعضای تیم نیز به درک مشترکی از آنها دست یافتند.
\end{itemize}
\subsection{نیازمندی‌های کارکردی}
\lr{-R1}
کندو باید فرآیند ثبت‌نام مقدماتی را فراهم سازد.

\begin{adjustwidth}{2em}{0pt}
\lr{-R1-1}
کندو باید در این مرحله امکان انتخاب دروس را فقط براساس نام درس به دانشجویان بدهد.
\\\lr{-R1-2}
کندو باید این امکان را داشته باشد که دانشجو براساس وضعیت تحصیلی خود در ترم جاری مشخص کند چند درصد احتمال دارد درسی را که انتخاب کرده در ترم آینده اخذ خواهد کرد.
\end{adjustwidth}
\lr{-R2}
کندو باید گزارشات ثبت نام مقدماتی را در اختیار معاونین آموزشی قرار دهد.
\begin{adjustwidth}{2em}{0pt}
\lr{-R2-1}
کندو باید تعداد افراد ثبت نام کرده در هر درس را مشخص کند و بتواند این اطلاعات را براساس سال ورودی و گرایش دانشجویان نیز تفکیک کند و به معاونین آموزشی ارائه بدهد.
                                                                                                                                                                                                                   \\\lr{-2-R2}                                                                                                                                                                                                                         کندو باید دراین مرحله امکان جست‌وجو و مرتب‌سازی را براساس درس و افراد ثبت نام کرده بدهد
                                                                                                                                                                                                                                                                                                                                                                                                                                                \end{adjustwidth}
                                                                                                                                                                                                                                                                                                                                                                                                                                                 \lr{-R3}                                                                                                                                                                                                                         کندو باید پس از دریافت دروس ارائه شده ترمی از معاون آموزشی لیست کامل دروس ترمی را در اختیار دانشجویان بگذارد. 
                                                                                                                                                                                                                                                                                                                                                                                                                                               \begin{adjustwidth}{2em}{0pt}
                                                                                                                                                                                                                                                                                                                                                                                                                                                 \lr{-R3-1}
                                                                                                                                                                                                                          کندو باید امکان فیلتر، جست‌وجو و مرتب‌سازی دروس ارائه‌شده براساس ویژگی های هردرس مانند: نام استاد، روز و ساعت برگزاری درس، تاریخ امتحان، تعداد واحد و... به دانشجو بدهد. 
                                                                                                                                                                                                                          \\\lr{-2-R3}                                                                                                                                                                                                                         کندو باید لیست دروس پاس‌نشده و قابل اخذ توسط دانشجو را از لیست دروس ارائه‌شده تفکیک کند و آن ها را به دانشجو نمایش دهد.  
                                                                                                                                                                                                                          \\\lr{-3-R3}
                                                                                                                                                                                                                         کندو باید واحدهای گذرانده‌نشده توسط دانشجو را با توجه به وضعیت تحصیلی او و رعایت شرایط پیشنیاز و همنیاز، برای دانشجو اولویت بندی کند (دروس با اولویت بالاتر دارای اهمیت بیشتری هستند) .
                                                                                                                                                                                                                                                                                                                                                                                                                                                                                                                                                                                                                                                        \end{adjustwidth}  
                                                                                                                                                                                                                                                                                                                                                                                                                                                                                                                                                                                                                                                                     \lr{-R4}                                                                                                                                                                                                                         کندو باید لیست دروس ارائه‌نشده در آن ترم و دلیل ارائه نشدن آن‌ها(مانند ترم نامرتبط، استقبال کم در ثبت‌نام مقدماتی، نبود استاد مناسب برای درس و...) را به دانشجو نمایش دهد. این کار با هدف اخذ درس به‌صورت معرفی به استاد و یا ارسال درخواست برای ارائه دروس مذکور انجام می‌گیرد.
\\\lr{-R5}                                                                                                                                                                                                                         کندو باید یک برنامه هفتگی پیشنهادی، خاص هر دانشجو بر اساس وضعیت کنونی و دروسی که بهتر است درترم پیش‌رو بردارد آماده کند
(این برنامه نتیجه‌ی نظر استاد راهنما، چارت پیشنهادی و وضعیت تحصیلی هر دانشجو می‌باشد)                                                                                                                                                                                                                                                                                                                                                                                                                                               . 
 \\\lr{-R6}                                                                                                                                                                                                                          کندو باید امکان ایجاد برنامه‌ی هفتگی از دروس ترمی ارائه‌شده را به دانشجویان بدهد.
                                                                                                                                                                                                                       \begin{adjustwidth}{2em}{0pt}
                                                                                                                                       \lr{-R6-1}
                                                                                                                                                                                                                        کندو باید امکان اخذ چندین برنامه ی هفتگی را به دانشجویان بدهد.                                                                                                                                                                                                                        
                                                                     \\\lr{-R6-2}
                                                                                                                                                                                                                        کندو باید در هنگام انتخاب هر درس از دانشجو سوال کند که این درس به کدام یک از برنامه های هفتگی دانشجو باید اضافه شود و برنامه ها را به صورت پویا به‌روزرسانی کرده و قابلیت نمایش آنها در هرلحظه را نیز داشته باشد.\\
                                                                                                                                                                                                                                                                                                                                                                                                                                               \lr{-R6-3}  
                                                                                                                                                                                                                        کندو باید درس های با اولویت بالاتر را برای اضافه شدن به برنامه هفتگی به دانشجو پیشنهاد دهد. هم‌چنین، در صورت عدم انتخاب این دروس توسط دانشجو اخطار مناسبی را به او نمایش دهد.\\
                                                                                                                                                                                                                         \lr{-R6-4}
                                                                                                                                                                                                                        کندو باید در صورت رخداد هرگونه تداخل هنگام انتخاب دروس از اخذ آن‌ها جلوگیری کند و در صورت وجود سکشن های دیگر با زمان مناسب، آن‌ها را به دانشجو پیشنهاد دهد.                                                                                                                                                                                                                         
\\\lr{-R6-5}                                                                                                                                                                                                                        کندو باید در صورت عدم رعایت شرایط پیش‌نیازها و هم‌نیازها توسط دانشجو از اخذ درس مربوطه جلوگیری کند و این امکان را به دانشجو بدهد که درخواستی جهت اخذ مجوز انتخاب درس به معاون آموزشی ارسال کند.
\\\lr{-R6-6}                                                                                                                                                                                                                         کندو باید امکان مقایسه بین برنامه های متعدد انتخاب‌شده توسط دانشجو را به او بدهد.
\\\lr{-R6-7}                                                                                                                                                                                                                          کندو باید ویژگی‌های هریک از برنامه‌های اخذشده توسط دانشجو مانند مجموع تعداد واحدها ، مجموع مدت زمان دروس، وضعیت امتحانات و... را به دانشجو نمایش دهد.                                                                                                                                                                                                    \\\lr{-R6-8}                                                                                                                                                                                                                         کندو باید تعداد افراد ثبت‌نام کرده در هر سکشن را به دانشجویان نشان دهد و مشخص کند که کدام یک از همراهان او سکشن مذکور را انتخاب کرده اند.                                                                                                                                                                                                                          \\\lr{-R6-9}                                                                                                                                                                                                                          کندو باید بر اساس سکشن‌ها و برنامه‌های اخذشده توسط دیگر دانشجویان و ظرفیت دروس ارائه‌شده، مشخص کند که هرکدام از سکشن‌ها و برنامه‌های هفتگی تنظیم‌شده توسط دانشجو چند درصد امکان اخذ دارند و درواقع رقابت برای اخذ آنها به چه صورت می‌باشد.                                                                                                                                                                                 \end{adjustwidth} 
\lr{-R7}                                                                                                                                                                                                                         کندو باید گزارش برنامه‌های هفتگی انتخاب‌شده توسط دانشجویان را که بخش اعظم آن تعداد افراد ثبت‌نام کرده در هر سکشن می‌باشد را در اختیار معاونین آموزشی قرار دهد.                                                                                                                                                                                                                         \begin{adjustwidth}{2em}{0pt}                                                                                                                                                                                                                         \lr{-R7-1}                                                                                                                                                                                                                         کندو باید ضمن تحویل گزارش، تحلیلی بر سایر گزارش‌ها را نیز در اختیار معاونین آموزشی قرار دهد. هم‌چنین باید در صورت لزوم نتایج این تحلیل‌ها را با نمودارهای مناسب (که منجر به سادگی و درعین‌حال کارآمدی بیشتر نتایج می‌شود)  نمایش دهد.
\\\lr{-R7-2}
          کندو باید امکان جست‌وجو، فیلتر کردن و مرتب‌سازی بر اساس فیلدهای مختلف (مثل: سال ورودی، گرایش، نام استاد، نام دروس، کد سکشن و...) را بر روی گزارش‌ها به معاونین آموزشی بدهد                                                                                                                                                                                                                .
\end{adjustwidth}
\lr{-R8}
                                                                                                                                                                                                                          کندو باید بخشی به نام اعلان‌ها داشته باشد که در آن  اطلاعیه‌ها و اخبار مربوط به جدیدترین رویدادهای آموزشی، تغییرات و مشکلات احتمالی سامانه قرار داده می‌شود.                                                                                                                                                                                                                        
\begin{adjustwidth}{2em}{0pt}
 \lr{-R8-1}
                                                                                                                                                                                                                          کندو باید برای هر یک از کاربران اعلان‌های مرتبط با او را در صفحه‌ی شخصی‌اش ظاهر کند. همچنین هر کاربر می‌تواند با رفتن به بخش اعلان‌های سایت، تمامی اعلان‌های موجود را مشاهده کند.
\\\lr{-R8-2}
         کندو باید اعلان‌های مربوط به تغییرات و یا مشکلات احتمالی سامانه را که مربوط به همه‌ی کاربران می‌شود در صفحه‌ی اصلی سایت و در هنگام ورود به تمامی 
        کاربران نمایش دهد.    
\end{adjustwidth}  
\hypertarget{one}{\lr{-R9}}
کندو باید دارای چندین بانک اطلاعاتی متفاوت باشد؛ محتوای این بانک‌ها عموماً به شکل خودکار از سیستم گلستان دریافت می‌شود و یا اطلاعات مربوطه به شکل دستی توسط کاربران  وارد سامانه می‌شود و سرانجام توسط سامانه در بانک‌ها طبقه‌بندی می‌شوند.
\begin{adjustwidth}{2em}{0pt}
\lr{-R9-1}
           کندو باید برای دانشجویان یک بانک اطلاعاتی داشته باشد که شامل اطلاعات (مانند نام، شماره‌ی دانشجویی، استاد راهنما و...) و وضعیت تحصیلی (مانند دروس گذرانده شده، نمرات کسب‌شده و...)         آنها می‌باشد که  این موارد در حساب کاربری هر دانشجو قرار دارد و برای او قابل‌مشاهده می‌باشد.
\\\lr{-R9-2}  
 کندو باید یک بانک اطلاعاتی اساتید داشته باشد که شامل نحوه‌ی ارزیابی (بارم‌بندی) هر استاد، نمونه جزوه، نتیجه‌ی ارزشیابی تحصیلی ترم‌های گذشته در قالب امتیاز و همچنین نظرات دانشجویان می‌باشد. این اطلاعات همگی در حساب کاربری هر استاد قرار دارد و برای سایرین قابل‌مشاهده است.
\\\lr{-R9-3}
          کندو باید برای رشته‌های تحصیلی موجود در دانشگاه یک بانک اطلاعاتی داشته باشد که اطلاعات آنها شامل سیلابس ارائه‌شده توسط وزارت علوم، شرایط اخذ دروس (پیش‌نیازها و هم نیازها)،چارت پیشنهادی و مراجع دروس مختلف برای هر رشته  می‌باشد. این اطلاعات در بخش رشته‌ها در سامانه قابل‌دسترسی می‌باشد.       
\end{adjustwidth}
\hypertarget{ww}{\lr{-R10}}
کندو باید دارای حساب‌های کاربری اختصاصی برای اشخاص مختلف باشد که این حساب‌ها به پنج نوع متفاوت تقسیم می‌شوند: دانشجویان، اساتید، مدیران سطح یک (ادمین‌های سیستم)، مدیران سطح دو (معاونین آموزشی) و افراد مهمان این پنج سطح را تشکیل می‌دهند. این دسته‌بندی به‌منظور مشخص کردن سطوح دسترسی متفاوت کاربران به کندو لحاظ شده است.
\begin{adjustwidth}{2em}{0pt}
\lr{-R10-1}
           کندو باید برای حساب‌های کاربری در سطح دسترسی مدیرسطح 1 یا ادمین سیستم، اجازه‌ی هرگونه تغییر، ایجاد و حذف در حساب‌های کاربری افراد مختلف، اضافه کردن اعلان جدید (شامل تغییرات و یا مشکلات احتمالی سامانه) و در صورت لزوم امکان ایجاد تغییرات دیگر در اطلاعات سایت را بدهد.     
\\\lr{-R10-2}
        کندو باید برای حساب‌های کاربری در سطح دسترسی مدیر سطح دو یا معاون آموزشی، اجازه‌ی وارد کردن، تغییر و حذف اطلاعات موجود در سایت، امکان اضافه کردن اعلان جدید، امکان مشاهده و تغییر مشخصات فردی دانشجویان (اجازه‌ی مشاهده برنامه‌های هفتگی انتخاب‌شده دانشجویان به همره مشخصات دانشجو توسط مدیر سطح دو فقط منوط به اجازه‌ی خود دانشجویان می‌باشد).امکان مشاهده و تغییر حساب‌های کاربری اساتید و درنهایت مشاهده‌ی گزارش‌های ثبت‌نام و دانلود آنها را داشته باشد.     
\\\lr{-R10-3}     
    کندو باید برای حساب کاربری دانشجو امکان تغییر مشخصات فردی، نظر دهی در مورد اساتید، دانلود جزوات اساتید از حساب کاربری آنها، دانلود اطلاعات آموزشی رشته تحصیلی خود (مثل سیلابس و چارت پیشنهادی و...) که در بخش رشته‌ها قرار دارد و هم‌چنین امکان دانلود برنامه‌های هفتگی تنظیم‌شده توسط دانشجو را داشته باشد.
\\\lr{-R10-4}
        کندو باید برای حساب کاربری استاد اجازه‌ی تغییر و اضافه کردن اطلاعات (به جز بخش امتیاز) در حساب کاربری همان استاد بدهد. هم‌چنین هر استاد باید امکان دانلود اطلاعات آموزشی مربوط به رشته‌ی تحصیلی در حال فعالیتش که در بخش رشته‌ها قرار دارد را داشته باشد.
\\\lr{-R10-5}
        کندو باید برای حساب‌های کاربری در سطح دسترسی مهمان، صرفاً اجازه‌ی مشاهده‌ی بعضی از بخش‌های مختلف را بدهد در ضمن این کاربران اجازه‌ی دانلود و یا تغییر هیچ کدام از بخش‌های سایت را نخواهند داشت.
                                                                                                                                     \end{adjustwidth}                                                                                                                                                                                                                        \lr{-R11}
                                                                                                                                                                                                                          کندو باید برای حساب‌های کاربری با سطح دسترسی دانشجو، استاد و معاون سطح دو بخشی با نام همراهان داشته باشد. انتخاب همراهان با ارسال  "درخواست همراهی"  توسط دارنده‌ی حساب، برای کاربر موردنظر صورت می‌گیرد و در صورت پذیرش درخواست، این همراهی شکل می‌گیرد. به‌صورت پیش‌فرض معاون سطح دو و استاد راهنمای مرتبط با هر دانشجو جز همراهان او می‌باشند. مزیت وجود همراهان این خواهد بود که دانشجو می‌تواند برنامه‌ی خود را با آنها به اشتراک بگذارد و از وضعیت ثبت‌نام همراهان خود در درس‌های مختلف مطلع شود و همچنین امکان چت کردن با آنها را نیز داشته باشد.
\\\lr{-R12}
 کندو باید امکان ایجاد ارتباط بین کاربران مختلف را به شکل‌های متفاوت از جمله ارسال درخواست، طرح پرسش و چت کردن  داشته باشد.
 \begin{adjustwidth}{2em}{0pt}                                                                                                                                                                                                                                                                                                                                                                                                                                                  \lr{-R12-1}
                                                                                                                                                                                                                          کندو باید اجازه چت کردن دانشجو را فقط با همراهانش بدهد، یعنی تنها ارتباطات میان دانشجو با استاد راهنما، دانشجو با دیگر دانشجویان همراه خود و دانشجو با معاون سطح دو مجاز می‌باشد. به همین منظور نیز استاد راهنما و معاون سطح دو به شکل پیش‌فرض در لیست همراهان دانشجویان قرار دارند.
\\\lr{-R12-2}
          کندو باید برخلاف چت کردن که دارای محدودیت بین کاربرهای مختلف می‌باشد امکان ارسال درخواست و یا طرح پرسش و سپس دریافت پاسخ را به‌تمامی کاربران درون سامانه بدهد.
\end{adjustwidth}          
\lr{-R13}                                                                                                                                                                                                                        
\hypertarget{oo}{کندو}
باید امکان ورود اطلاعات را هم به شکل دستی (توسط معاونین سطح یک و دو) و هم به شکل خودکار (از سایر پایگاه‌های داده) داشته باشد.
\begin{adjustwidth}{2em}{0pt} 
\lr{-R13-1}
کندو باید بتواند با دریافت فایل‌های ورودی با قالب‌های رایج مثل 
\lr{xml}
 یا 
 \lr{csv}
  اطلاعات را به صورت خودکار در بانک‌های اطلاعاتی خود ذخیره کند.
\end{adjustwidth} 
\lr{-R14}
کندو باید امکان دانلود کردن فایل‌ها تحت قالب‌هایی هم‌چون  
\lr{pdf, docx, csv, xlsx}
 به آن سطوح ازحساب‌های کاربری که اجازه‌ی دانلود محتوا دارند را فراهم آورد.
 
\subsection{نیازمندی‌های غیرکارکردی}
\subsubsection*{\lr{ -R1}نیازمندی‌های کارایی}

\begin{adjustwidth}{2em}{0pt}
\lr{-R1-1}                                                                                                                                                                                                                         
کندو باید پس از ارسال درخواست از سمت کاربر بین 1 تا 3 ثانیه به آن پاسخ دهد.
\\\lr{-R1-2}                                                                                                                                                                                                                         
کندو باید در به کارگیری از منابع سرور بهترین عملکرد و کارایی را داشته باشد.
\\\lr{-R1-3}                                                                                                                                                                                                                         
کندو باید آمادگی پاسخگویی به ده هزار کاربر در ثانیه را داشته باشد و با افزایش تعداد کاربران کوچک‌ترین خللی در آن به وجود نیاید؛ باید بدون معطلی درخواست‌ها را پردازش کرده و پاسخ دهد.
\\
\end{adjustwidth}
\subsubsection*{\lr{ -R2}نیازمندی‌های کیفیت}
\begin{adjustwidth}{2em}{0pt}
\lr{-R2-1}
کندو باید در 99 درصد مواقع پاسخگوی درخواست‌های کاربران باشد و از دسترس کاربران خارج نشود؛ در صورت بروز چنین اتفاقی صفحه‌ی اول سایت باید تغییر شکل داده و پیغام مناسبی را از علت خطا نمایش دهد و تخمین زمانی که سایت به حالت عادی بازخواهد گشت نیز در آن نوشته شود.
\\\lr{-R2-2}
کندو باید همواره گزارش آخرین اتفاقات و رخدادهای سیستم (وضعیت کلی سیستم) را در هر دو سرور اصلی و ثانویه ذخیره کند تا درصورتی‌که حادثه‌ای ناگهانی پیش آمد، امکان بازیابی به یکی از وضعیت‌های ایده آل قبلی وجود داشته باشد. این وضعیت‌ها در سرور اصلی به مدت یک هفته و در سرور ثانویه به مدت یک ماه نگهداری می‌شوند. (شایان ذکر است که همواره ایده‌آل‌ترین وضعیت در هرماه که عموماً روزهای آخر آن ماه است، قبل از پاک‌سازی سرور ثانویه در حافظه‌ی دائمی هر دو سرور با عنوان وضعیت ایده آل ماهx  ذخیره شده و به مدت یک سال در آنجا نگهداری می‌شود).
\\
\end{adjustwidth}

\subsubsection*{\lr{ -R3}نیازمندی‌های ایمنی}
\begin{adjustwidth}{2em}{0pt}
\lr{-R3-1}
کندو باید از ورود به حالت ناخواسته جلوگیری کند و درصورتی‌که چنین حالتی رخ داد با استفاده از وضعیت‌های سیستم که در سرورهای خود ذخیره می‌کند، به‌سرعت مشکل را حل کرده و به شرایط عادی بازگردد.
\\
\end{adjustwidth}
\subsubsection*{\lr{ -R4}نیازمندی‌های امنیت}
\begin{adjustwidth}{2em}{0pt}
\lr{-R4-1}
کندو باید درخواست‌هایی که ارسال و دریافت می‌کند را به شکل رمزنگاری‌شده
\LTRfootnote{Encrypted}
 انتقال دهد، این موضوع نیازمند آن است که برای وب‌سایت گواهی امنیتی 
\lr{SSL} 
\LTRfootnote{Secure Sockets Layer}
 تهیه شود تا اطلاعات به کمک رمزنگاری دو طرفه ردوبدل شوند.
\\\lr{-R4-2}
کندو باید داده‌هایی که مربوط به نام کاربری و رمز عبور کاربران برای ورود به سایت هستند را به شکل رمزنگاری‌شده در پایگاه داده‌ی خود ذخیره کند.
\\\lr{-R4-3}
کندو باید برای جلوگیری از نفوذ به سرور اصلی از یک فایروال بین شبکه‌ی داخلی و شبکه‌ی جهانی اینترنت استفاده کند تا درخواست‌هایی که به سمت وب‌سرور ارسال می‌شوند، اول از همه توسط فایروال بررسی شوند و در صورت صحت درخواست‌ها، اجازه‌ی ارسال آنها به سمت وب‌سرور داده شود.
\\\lr{-R4-4}
کندو باید به‌گونه‌ای پیاده‌سازی شود که در صورت نفوذ افراد مخرب به وب‌سرور، امکان دسترسی به پایگاه داده کندو وجود نداشته باشد؛ برای دستیابی به این امر باید در شبکه‌ی داخلی کندو از یک فایروال محلی
\LTRfootnote{Local Firewall}
 استفاده شود تا درخواست‌هایی که بین اجزای مختلف (زیرسیستم‌ها) کندو جابه‌جا می‌شوند (به‌خصوص کوئری های ارسال‌شده از سمت وب‌سرور برای پایگاه داده) قبل از هر چیز توسط این فایروال محلی پایش شوند و در صورت نبود مشکل، درخواست‌ها انتقال یابند.
\\\lr{-R4-5}
کندو باید این قابلیت را داشته باشد که اگر میزان درخواست‌های تکراری و نامعقول از پنج هزار درخواست در ثانیه بیشتر شد، بسته‌های مشابه بعدی همگی 
\lr{drop}
 شده و 
 \lr{ip}
 هایی که این درخواست‌ها را ارسال کرده‌اند نیز مسدود شوند، درنهایت هم اعلانی برای ادمین سیستم به همراه شرح اتفاقاتی که رخ داده است ارسال شود.
\\\lr{-R4-6}
 کندو باید سیاست‌هایی که در بخش 
 \hyperlink{ww}{\underline{\lr{R10}}} 
 از نیازمندی‌ها به آن‌ها اشاره شده است و مربوط به ارتباط کاربران با سطوح دسترسی متفاوت به سامانه هستند را به شکل بسیار سخت‌گیرانه‌ای اجرا کند. در راستای همین سیاست‌گذاری، امکان تغییر سطوح دسترسی کاربران، فقط توسط تیم توسعه قابل انجام می‌باشد و حتی ادمین های سیستم نیز اجازه‌ی تغییر آن‌ها را ندارند.
\\\lr{-R4-7}
کندو باید به‌محض آن‌که کاربری به سامانه وارد شد
\LTRfootnote{Login}
 و وارد حساب کاربری خود شد یک ایمیل یا یک پیامک (یا هر دو) را برای صاحب حساب کاربری مربوطه رسال کند. در پیام ارسال‌شده باید زمان ورود به حساب، نام دستگاه و ipای که با آن درخواست ورود ارسال‌شده است ذکر شود. هم‌چنین پیام باید حاوی یک لینک تخریب نشست باشد که چنانچه ورود غیرمجاز توسط صاحب حساب تشخیص داده شد، بتوان از راه دور این نشست را از بین برد.
\end{adjustwidth}

\subsection{قیود طراحی}
\subsubsection{محدودیت سازگاری}
\hypertarget{gg}{سامانه‌ی کندو}
باید توسط ادمین های سیستم نگهداری و مدیریت شود، لذا نیاز است که این افراد مهارت‌های لازم جهت کار با سیستم و نگهداری آن را داشته باشند. با توجه به همین موضوع و نیز ذکر این نکته توسط مشتری که ادمین های سامانه کندو همان ادمین های سیستم گلستان هستند، محدودیتی در نظر گرفته شد که طبق آن سامانه‌ی کندو حتماً باید بر روی سرور ویندوزی نصب شود، وب‌سرور سامانه 
\lr{IIS}
\LTRfootnote{Internet Information Services}
 و زبان سمت سرور
\lr{asp.net}
درنظر گرفته شود و پایگاه داده‌ی کندو نیز
\lr{SQL Server}
ماکروسافت باشد تا سازگاری کاملی بین این سامانه و سیستم گلستان ایجاد شود، و هم‌چنین ادمین های سیستم گلستان بتوانند با یک آموزش مختصر، سامانه‌ی کندو را نیز در کنار سیستم گلستان پایش و مدیریت کنند.

\subsubsection{محدودیت طراحی واسط کاربری}
واسط کاربری سامانه باید به صورت واکنش‌گرا 
\LTRfootnote{Responsive}
طراحی شود.\\
سامانه‌ی کندو باید نمایش مناسبی در تمامی مرورگرهای رایج داشته باشد، در واقع باید قابلیت حمل داشته باشد و با اکثر مرورگرها سازگاری 
\LTRfootnote{compatibility}
 داشته باشد.

\subsection{صفت های ‌سیستم نرم‌افزاری}
\subsubsection{ماژولار بودن سیستم}
یکی از قابلیت‌هایی که در سامانه‌ی کندو لحاظ شده است، ماژولار بودن بخش‌های مختلف سیستم می‌باشد؛ به این صورت که هر کدام از بخش‌های سیستم دارای یک ماژول مجزا می‌باشند و همگی این ماژول‌ها با یکدیگر در ارتباط هستند. به واسطه‌ی این ویژگی، تیم توسعه قادر خواهد بود که بخش‌های مختلف سیستم را به‌صورت مجزا ایجاد کند، از میان بردارد، توسعه بدهد و عیب‌یابی کند؛ بدون آن‌که بخش‌های دیگر سیستم تحت‌الشعاع این تغییرات قرار بگیرند یا خللی در اصل سیستم ایجاد شود. در ضمن، امکان استفاده از کل و یا قسمتی از هر ماژول در پروژه‌ها و نسخه‌های بعدی این سامانه و یا سیستم‌های دیگری که ممکن است در آینده پیاده‌سازی شوند نیز وجود دارد. شایان توجه است که ماژولار بودن سیستم با فرآیند چابکی که تیم توسعه در تحلیل و طراحی خود دنبال می‌کند نقاط اشتراک بسیاری (ازجمله منعطف بودن در برابر تغییرات) را داراست.
\subsubsection{دو وجهی بودن سیستم (خودکار - غیرخودکار) }
یکی دیگر از قابلیت‌های سامانه‌ی کندو، خودکار بودن فرآیندهای مختلف آن می‌باشد؛ فرآیندهایی مانند پردازش داده‌های سیستم جهت ارائه‌ی برنامه‌ی پیشنهادی، تحلیل برنامه‌های دانشجویان و تحویل گزارش‌های آنها به معاونین آموزشی، بررسی تداخل و روابط پیش‌نیازی و هم‌نیازی به کمک داده‌های درون سیستم، خروج و حتی دریافت اطلاعات که در بخش
\hyperlink{oo}{\underline{\lr{R13}}} 
از نیازمندی‌های کارکردی اشاره شد. تمامی موارد قبل نشان از این دارند که فرآیندهای سامانه‌ی کندو به شکل کاملاً خودکار صورت می‌گیرند، البته ذکر این نکته نیز ضروری است که سامانه کندو این امکان را فراهم می‌سازد که فرآیندهایش به شکل غیر خودکار نیز صورت بگیرند، یعنی به کاربران خود اجازه می‌دهد که خودشان مدیریت اطلاعات سامانه را به عهده بگیرند. همین موضوع است که باعث می‌شود دانشجو بتواند برنامه‌های مطلوب خود را تنظیم کند، معاون آموزشی بتواند گزارش برنامه‌ها را  از سامانه‌ی کندو دریافت کند، تحلیل بر روی داده‌ها را انجام دهد  یا این‌که دریافت اطلاعات توسط سامانه  کندو می‌تواند به شکل دستی و توسط کاربرانی که دسترسی مربوطه را دارند صورت بگیرد. همگی این موارد به این موضوع اشاره دارند که سامانه‌ی کندو باید رویکرد غیر خودکار نیز داشته باشد.
لذا این نکته نتیجه می‌شود که سامانه‌ی کندو یک سیستم دو وجهی می‌باشد که هم قادر است به شکل خودکار فرآیندها را صورت دهد و هم می‌تواند به‌عنوان ابزاری عمل کند که کاربرانش بتوانند به کمک قابلیت‌های تعبیه‌شده در آن، اطلاعات مختلف فرایند انتخاب واحد را بررسی نمایند، تغییرات خود را روی آنها اعمال کنند و درنهایت نتایج حاصل را مشاهده کنند.

\clearpage

\chapter{مدل‌سازی دامنه}
\section{جمع‌آوری اطلاعات دامنه کاربرد}
خروجی این گام اطلاعات و مستندات مربوطه در زمینه مورد کاربرد می‌باشد که از آن‌ها برای کشف مفاهیم مربوطه در دامنه‌ی کاربرد و درنهایت به تصویر کشیدن این مفاهیم توسط نمودار کلاس استفاده می‌شود. در این گام از شرح نیازمندی‌های نرم‌افزار که در فصل قبل جمع‌آوری‌شده است به‌عنوان اطلاعات و مستندات مربوطه در زمینه کاربرد می‌شود.
\section{طوفان فکری}
در این گام طبق نتایج گام قبل که همان اطلاعات و مستندات مربوطه در زمینه کاربرد می‌باشد، مفاهیم مهم و اصلی دامنه کاربرد شناسایی و فهرست شده‌اند.
\section{دسته‌بندی نتایج طوفان فکری}
پس از شناسایی مفاهیم مهم دامنه کاربرد از قوانین دسته‌بندی برای دسته‌بندی نتایج طوفان فکری استفاده می‌شود. این مفاهیم به کلاس‌ها، ویژگی‌ها، مقادیر ویژگی‌ها، و روابط دسته‌بندی شدند. لازم به ذکر است برخی از این مفاهیم پس از طراحی مدل دامنه و با تصمیم اعضای گروه از این دسته‌بندی حذف شده و مفاهیم مناسب‌تری جایگزین آن‌ها شده است. مفاهیم به‌ دست آمده از طوفان فکری در جدول زیر دسته‌بندی شده‌اند:

\includepdf[page=-]{table}

\clearpage
\section{به ‌تصویر کشیدن مدل دامنه}  

در این گام مطابق با نتایج دسته‌بندی‌شده در جدول قبل، مدل دامنه به تصویر کشیده شده است. برای این کار از نمودار کلاس استفاده‌شده است.
شکل زیر تصویر نمودار کلاس طراحی‌ شده را نشان می‌دهد:

\begin{sidewaysfigure}[h!]
  \includegraphics[width=\linewidth]{ClassDiagram-final.jpg}
  \caption{نمودار کلاس مدل دامنه}
  \label{fig:1}
\end{sidewaysfigure}

\clearpage



\section{مرور مدل دامنه}
در نمودار طراحی‌ شده برخی از مفاهیم و روابط که ممکن بود منجر به خطاهای احتمالی شوند و یا با اطلاعات به‌دست آمده قبلی مغایرت داشتند اصلاح شدند.


در رابطه با نمودار مدل دامنه موارد زیر باید در نظر گرفته شوند:
\begin{itemize}
\item
در نمودار نهایی برای درک بهتر روابط و مفاهیم موجود در نمودار، کلاس‌های هر مجموعه خاص به یک رنگ در آمده‌اند. کلاس‌های مربوط به اخطار به رنگ قرمز، کلاس‌های مربوط به گزارش به رنگ سبز چمنی، کلاس‌های مربوط به برنامه به رنگ نارنجی، کلاس‌های انجمنی به رنگ بنفش، کلاس‌های کاربران به رنگ سبز کم‌رنگ، و مابقی کلاس‌ها به رنگ آبی نشان داده‌ شده‌اند.

\item
در برخی از روابط برای نشان دادن تعداد از اعداد 
\lr{m}
 و 
 \lr{n}
  استفاده شده است که علت آن مشخص نبودن تعداد دقیق این اشیا بوده است. برای مثال می‌دانیم که تعداد دانشجویان نمی‌تواند صفر باشد و همچنین باید یک محدودیت هم داشته باشد اما تعداد دقیق آن‌ها نیز در دسترس نبود. بنابراین برای نشان دادن تعداد آن‌ها از بازه‌ی 
  \lr{m..n}
   استفاده شده است.

\item
یکی از مفاهیمی که در شرح نیازمندی‌های نرم‌افزار و همچنین طوفان فکری ذکر شده است قابلیت فیلتر و جستوجوی بخش‌های مختلف سامانه است که در نمودار مدل دامنه در قالب رابطه مشاهده کردن گنجانده‌ شده است. بنابراین برای اینکه تعداد روابط بیش‌ از حد نشود از نوشتن رابطه‌ی جداگانه برای این موارد اجتناب شده است.
\end{itemize}

\clearpage
\chapter{طراحی معماری}
\section{تعیین اهداف معماری}
یکی از اهداف طراحی سامانه تنظیم برنامه درسی، تسهیل روند انتخاب واحد، و اخذ دروس برای دانشجویان و
آسان سازی مدیریت و ایجاد تغییرات برای مدیران سطح 1 و 2 می‌باشد. لذا در جهت بهبود هر چه بیشتر
سامانه بر آن شدیم که معماری مشخصی برای سامانه در نظر بگیریم که با توجه به قوانین حاکم بر معماری
روند افزایش کارایی سامانه، اصلاح، و ایجاد تغییرات در آن با سهولت بیشتری صورت گیرد.

\section{تعیین نوع سیستم}
سامانه کندو به دلیل وجود ویژگی‌هایی زیر زیر یک سیستم تعاملی محسوب می‌شود.

\begin{enumerate}
\item
تعامل بین کندو و کاربر برای انجام دنباله‌ای از درخواست‌های کاربر.
\item
کاربر خدماتی را از کندو درخواست نموده و سیستم نتایج عملیات را برای او نمایش می‌دهد.
\item
حالت سیستم، پیشرفت فرایند مد نظر را منعکس می‌کند.
\end{enumerate}

همچنین از آنجایی که کندو یک سیستم نرم‌افزاری مبتنی بر وب می‌باشد که همانند بسیاری از سیستم‌های دیگر با کاربر در تعامل می‌باشند، مناسب‌ترین سیستمی که می‌تواند بیان‌کننده‌ی معماری کندو باشد، سیستم تعاملی می‌باشد.

\section{به‌ کارگیری یک سبک معماری}
با توجه به آن که کندو یک سیستم تعاملی می‌باشد، مناسب‌ترین معماری‌ای که می‌توان برای آن در نظر گرفت معماری لایه‌ای می‌باشد.
شکل زیر معماری 6 لایه‌ای سامانه کندو را نشان می‌دهد که در ادامه آن توضیح مختصری درمورد هر لایه آن آورده شده است.

\begin{figure}[h!]
  \includegraphics[width=\linewidth]{MVC.jpg}
  \caption{معماری 6 لایه سامانه کندو}
  \label{fig:1}
\end{figure}

\begin{enumerate}
\item
لایه رابط گرافیکی کاربر
\LTRfootnote{GUI layer} 
:
در لایه اول رابط گرافیکی و ظاهر سامانه پیاده سازی می‌شود. دراین لایه اطلاعاتی که کاربر برای انجام عملیات‌های خود به سامانه کندو وارد می‌کند، دریافت می‌شود و سپس این اطلاعات به لایه کنترلر منتقل می‌شود. همچنین نتایج عملیات‌هایی که کاربر انجام داده است در این لایه برای او نمایش داده می‌شود.
\item
لایه کنترلر
\LTRfootnote{Controller layer} 
:
برای تفکیک بخش ظاهری و درونی سیستم از لایه‌ای میانی به نام کنترلر استفاده می‌شود که وظیفه پل زدن بین این دو بخش را ایفا می‌کند. هدف در این لایه پیاده‌سازی 
\lr{API}
\LTRfootnote{Application Programming Interface} 
 مناسب بدون وابستگی به شیوه انجام عملیات در لایه سرویس، و شیوه نمایش اطلاعات در لایه رابط گرافیکی است. این تفکیک سبب ساده‌سازی توسعه و نگهداری سیستم می‌شود.
\item
لایه سرویس
\LTRfootnote{Service layer} 
:
در این لایه که هسته مرکزی سامانه است و شامل مهمترین زیرسیستم‌های سامانه می‌شود، منطق سامانه کندو پیاده‌سازی می‌شود که معماری داخل آن به تفصیل در بخش معماری لایه سرویس بیان شده است. لازم به ذکر است که در این لایه از هرگونه انجام مستقیم عملیات در پایگاه داده یا ارتباط مستقیم با شبکه یا رابط کاربری باید پرهیز شود و صرفا به پیاده‌سازی منطق سامانه در این لایه پرداخته شود.
\item
لایه مخزن
\LTRfootnote{Repository layer}  
:
در لایه مخزن تمامی ارتباطات و عملیات مورد نیاز در کار با پایگاه داده یا شبکه پیاده‌سازی می‌شود تا در لایه سرویس از آن‌ها استفاده شود. این جداسازی سبب می‌شود وابستگی لایه سرویس به معماری لایه‌های پایگاه داده و شبکه به حداقل میزان ممکن برسد و تغییر احتمالی در معماری آن لایه‌ها، ساده‌تر صورت پذیرد.
\item
لایه پایگاه داده
\LTRfootnote{Database layer} 
:
در این لایه پایگاه داده قرار دارد که برای نگهداری و ذخیره اطلاعات سامانه کندو از آن استفاده می‌شود. برای نوشتن یا خواندن این اطلاعات در لایه سرویس الزاماً باید از دستورات پیاده سازی شده در لایه مخزن استفاده شود و نمی‌توان به طور مستقیم از این لایه در لایه سرویس استفاده کرد.
\item
لایه ارتباطات شبکه
\LTRfootnote{Network connection layer} 
:
از آنجایی که سامانه ما نیازمند اطلاعات سایر سامانه‌ها مانند سامانه گلستان و سامانه ثانیه می‌باشد این لایه برای برقرای ارتباط آن‌ها در نظر گرفته شده که تنها با لایه مخزن در ارتباط است و اطلاعات ارسالی به سامانه‌های دیگر از طریق لایه مخزن به لایه شبکه می‌رسد. همچنین پس از دریافت اطلاعات سامانه‌های دیگر، از طریق لایه مخزن این اطلاعات به لایه پایگاه داده منتقل می‌شوند. بنابراین لایه پایگاه داده به طور غیر مستقیم با لایه شبکه مرتبط است تا با تغییر در سامانه‌های دیگر تنها تغییرات مختصری در لایه مخزن برای تبادل آن‌ها صورت گیرد و نیازی به تغییرات اساسی در لایه پایگاه داده نباشد.

\end{enumerate}

\section{تعیین عملیات، واسط‌ها، و عملیات زیرسیستم‌ها}

\subsection{معماری لایه سرویس}
در لایه سرویس، معماری به صورت رویداد-رانده پیاده‌سازی شده است. هسته نقش مدیریت دیگر زیرسیستم‌ها را برعهده دارد و ارتباط آن‌ها را با یکدیگر ممکن می‌سازد. هفت زیرسیستمی که توسط این لایه مدیریت می‌شوند عبارتند از:

\begin{enumerate}
\item
احراز هویت: 
کاربر قبل از فعالیت در سیستم ابتدا باید احراز هویت شود تا نقش او در سیستم، اطلاعات مرتبط با او، و میزان دسترسی‌هایش مشخص گردد.

\item
ثبت‌نام مقدماتی: 
این زیرسیستم ثبت‌نام مقدماتی را پیاده‌سازی می‌کند و پس از دریافت اطلاعات از کاربر نتایج را به هسته برمی‌گرداند تا برای لایه شبکه و در نهایت سیستم گلستان ارسال کند.

\item
همراهان: 
زیرسیستم همراهان امکان افزودن یا حذف همراهان را برای هر کاربر ممکن می‌سازد. همچنین امکان اشتراک‌گذاری برنامه درسی نیز بین آن‌ها در نظر گرفته شده است.

\item
پرسش و پاسخ: 
دانشجو در این بخش می‌تواند از طریق چت با همراهان خود، اساتید، و معاون آموزشی در ارتباط باشد. همچنین در بخش سوالات متداول می‌تواند از سوالاتی که به طور مکرر پرسیده شده اطلاع یابد تا از پرسش و پاسخ تکراری جلوگیری شود.

\item
تنظیم برنامه درسی: 
زیرسیستم تنظیم برنامه‌ی درسی اصلی‌ترین زیرسیستم ما از لحاظ عمللکرد محسوب می‌شود، که به وسیله‌ی آن انجام فرایندهای مختلفی اعم از نمایش دروس به روش‌های مختلف، امکان چینش دروس توسط دانشجویان، ارائه‌ی برنامه‌ی پیشنهادی خاص هر دانشجو، کنترل انواع خطاها، و در نهایت تحلیل برنامه‌های انتخابی پیاده‌سازی شده توسط دانشجویان می‌توان اشاره کرد، که همه‌ی این موارد توسط این زیرسیستم فراهم شده‌اند و قابل استفاده می‌باشند.

\item
اعلانات: 
این بخش امکان نمایش اخبار و اطلاعیه‌های موجود در سایت را میسر می‌سازد، همچنین وظیفه دارد اخبار مرتبط با مشکلات احتمالی سامانه و اطلاعیه‌‌‌های مهم که به تمامی دانشجویان مربوط می‌شود را در صفحه‌ی ورود به سایت قرار دهد، و در نهایت اخبار و اطلاعیه‌های مختص به هر دانشجو را در صفحه‌ی کاربری او نمایش دهد. در ضمن بخش اعلان‌ها در سامانه کندو به طور کامل توسط این سیستم فراهم و قابل استفاده می‌باشد.

\item
گزارش: 
این بخش امکان تحلیل و آماده‌سازی گزارشات مختلف بر روی فرایندهای ثبت‌نام مقدماتی و برنامه‌های هفتگی پیاده‌سازی شده توسط دانشجویان را میسر می‌سازد، هنگامی که معاون آموزشی می‌تواند گزارشی از وضعیت ثبت‌نام دانشجویان، مانند میزان تمایل دانشجویان به اساتید مختلف و یا سکشن های خاص، ساعت‌هایی که تمایل بیشتری در بین دانشجویان دارد و دبگر موارد مشابه را مشاهده کند، قابلیت‌هایی است که توسط این بخش از زیرسیستم به او داده شده است.


\end{enumerate}

در شکل زیر معماری این لایه نشان داده شده‌ است.

\begin{figure}[h!]
  \includegraphics[width=\linewidth]{ServiceLayer.jpg}
  \caption{معماری سیستم رویداد-رانده لایه‌ی سرویس}
  \label{fig:1}
\end{figure}


\subsection{معماری زیرسیستم تنظیم برنامه درسی}
سیستم تنظیم برنامه درسی به صورت رویداد-رانده و با یک کنترلگر مرکزی پیاده سازی می‌شود که پنج زیرسیستم زیر را ساماندهی می‌کند:

\begin{enumerate}
\item
 تنظیم برنامه دستی: 
کاربر می‌تواند به طور دستی برنامه‌های خود را با توجه به دروس ارائه شده بچیند و تا رسیدن به حالت مطلوب خود در آن تغییرات مدنظر خود را اعمال کند.

\item
پیشنهاد برنامه: 
درصورت تمایل دانشجو سامانه میتواند با توجه به اطلاعات تحصیلی او و دروس ارائه شده برنامه‌ای پیشنهاد دهد که دانشجو از آن بدون تغییر یا با تغییرات مدنظر خود استفاده کند.

\item
درخواست مجوز: 
در صورت لزوم به اخذ مجوز آموزشی دانشجو می‌تواند مجوز مورد نظر خود را به معاون آموزشی درخواست دهد.

\item
تحلیلگر برنامه: 
کاربر می‌تواند اطلاعات مفیدی درباره برنامه‌های تنظیم شده کنونی خود مانند تعداد واحد اخذشده در برنامه،‌ درصد تقریبی احتمال اخذ آن در روز انتخاب واحد، روزهای سنگین‌تر و سبک‌تر برنامه و موارد دیگر را مشاهده کند و از این طریق برنامه های خود را به راحتی مقایسه کند و آن ها را بهبود ببخشد.

\item
مدیریت کننده خطا: 
در صورت وجود خطا در برنامه فعلی، خطای مربوطه به اطلاع دانشجو خواهد رسید. برنامه‌های دارای خطا امکان اخذ در انتخاب واحد را ندارند.


\end{enumerate}

در شکل زیر معماری این زیرسیستم نشان داده شده ‌است.

\begin{figure}[h!]
  \includegraphics[width=\linewidth]{scheduler.jpg}
  \caption{معماری سیستم رویداد-رانده زیرسیستم تنظیم برنامه درسی}
  \label{fig:1}
\end{figure}

\subsection{معماری زیرسیستم کنترل خطا}
زیرسیستم کنترل خطا نیز از یک معماری رویداد-رانده پیروی می‌کند و شامل یک کنترل مرکزی و بخش‌های زیر می‌باشد:

\begin{itemize}
\item
 کنترل کننده تداخل زمانی

\item
کنترل کننده حداکثر واحد قابل اخذ

\item
کنترل کننده حداقل واحد قابل اخذ

\item
کنترل کننده رعایت هم‌نیاز

\item
کنترل کننده رعایت پیشنیاز

\item
کنترل کننده رعایت متضاد

\item
کنترل کننده رعایت جنسیت

\item
کنترل کننده عدم اخذ تکراری درس 
\end{itemize}

در زیر نمودار مربوط به این زیرسیستم نشان داده‌ شده‌ است.

\begin{figure}[h!]
  \includegraphics[width=\linewidth]{WarningController.jpg}
  \caption{معماری سیستم رویداد-رانده زیرسیستم کنترل کننده خطا}
  \label{fig:1}
\end{figure}

\section{اعمال قوانین طراحی نرم‌افزار‬}
\subsection{طراحی برای تغییر}
رویدادهای مختلفی در سیستم درنظر گرفته شده‌اند، که با توجه به ماژولار بودن سیستم، هر کدام از این بخش ها از یکدیگر مجزا شده‌اند و توسط زیرسیستم‌های متفاوتی پیاده سازی شده‌اند و این موضوع باعث شده است که در این سیستم وابستگی بخش های مختلف به هم از بین برود و امکان تغییر و بروزرسانی بخش های مختلف، بدون نگرانی از تغییر در بقیه ی بخش‌ها برای تیم توسعه فراهم شده باشد.
\subsection{جداسازی دغدغه‌ها}
تمرکز یک‌باره و همزمان به همه‌ی جنبه‌های سیستم تنظیم برنامه درسی باعث ایجاد مشکلات متعددی در مراحل
پیاده‌سازی پروژه می‌شود. جداسازی دغدغه‌ها مسئله‌ی طراحی نرم‌افزار را به دو سطح تقسیم می‌کند. در سطح
بالاتر چگونگی انجام فرایند کلی طراحی و در سطح پایین‌تر طراحی اجزا و مولفه‌های سیستم می‌باشد.
جداسازی دغدغه‌ها، راهنمایی برای ارضای نیازمندی‌ها می‌باشد.
\subsection{پنهان‌سازی اطلاعات}
به دلیل وجود معماری چند لایه در سیستم کندو کاربر سیستم فقط با لایه رابط گرافیکی سیستم تعامل دارد که این
موضوع باعث می‌شود کاربر سیستم وارد جزئیات سیستم نشود. در سیستم کندو اطالعات مهمی از جمله
اطالعات خصوصی دانشجویان وجود دارد که به دلیل امنیت نرم‌افزار باید در امان باشد که در لایه پایگاه داده
ذخیره شده است و از دسترسی به دور می‌باشد.
\subsection{چسبندگی زیاد}
نیازمندی‌های در نظر گرفته شده برای هر بخش فقط توسط زیرسیستم مربوط به آن بخش پیاده‌سازی شده است که این موضوع مستقل بودن زیر‌سیستم‌های مختلف از یکدیگر را نشان می‌د‌هد، و این نکته منجر به آن می‌شود که هر بخش تنها وظایف مربوط به خود را انجام دهد.
\subsection{جفت شدگی کم}
همانگونه که در بخش های قبلی ذکر شد هر لایه فقط وظیفه خود را انجام می‌دهد
 و اشتراکی در بین لایه‌ها وجود ندارد.
\subsection{جمع بندی}
برای طراحی معماری سیستم ابتدا اهداف تعیین معماری سیستم
مشخص گردید. سپس با انتخاب سبک معماری از معماری‌های موجود در مخزن معماری و تعیین نوع سیستم
کندو و رسم نمودار لایه‌ای آن، به مرحله ی تعیین عملیات، واسط‌ها و عملیات زیرسیستم‌ها رسیدیم. در این
مرحله نیز با تعیین هر کدام از زیرسیستم‌ها و نوع آن‌ها، و همچنین رسم نمودار زیرسیستم تنظیم برنامه درسی
و کنترل خطا از آن، تعیین معماری سیستم کندو به طور کامل و مشخص انجام
گردید.

\part{مدل‌سازی و طراحی سیستم‌های تعاملی}

\section*{اصلاحات فصول قبل}

پیش از شروع فصل چهارم، تغییراتی که تیم توسعه بنا به اصول چابکی در برخی از بخش‌های قبلی پروژه لحاظ کرده‌اند به شرح زیر بیان می‌شود:

- تغییراتی که در بخش واسط‌های حافظه اعمال شده است:

همان‌گونه که در بخش واسط‌های حافظه بیان شد، سامانه‌ی کندو نیازمند فضایی برای ذخیره‌سازی اطلاعات است. تیم توسعه در نظر دارد به هر دانشگاهی که جزو کاربران سامانه‌ی کندو می‌باشد به اندازه‌ی ۵ ترابایت فضا اختصاص دهد و برای هر دانشجو نیز فضایی با حجم حداکثر ۲۰۰ مگابایت در نظر گرفته شده است؛ این میزان فضا می‌تواند به طور تقریبی تعداد ۲۵۰۰۰ دانشجو را پشتیبانی کند. با توجه به این که به طور متوسط در هر سال تحصیلی در حدود ۱۷۰۰۰ دانشجو در هر کدام از دانشگاه‌های جامع کشور تحصیل می‌کنند، فضای باقی‌مانده به معاونین و اساتید دانشگاه اختصاص خواهد یافت. هم‌چنین، شایان ذکر است که اگر دانشگاه مربوطه از تیم توسعه درخواست فضای ذخیره‌سازی بیشتری داشته باشد، تیم توسعه با دریافت هزینه‌ای مشخص به ازای هرگیگابایت توانایی افزایش حجم برای دانشگاه و دانشجویان مشغول به تحصیل در آن را خواهد داشت.

- تغییراتی که در بخش نیازمندی‌های کارکردی صورت گرفته است به شرح زیر می‌باشد:

\lr{-R4}
کندو باید لیست دروس ارائه‌نشده در آن ترم را به دانشجو نمایش دهد. این کار با هدف اخذ درس به صورت معرفی به استاد و یا ارسال درخواست برای ارائه‌ی دروس مذکور انجام می‌گیرد.

\lr{-R6}
کندو باید امکان ایجاد و ویرایش برنامه‌های‌هفتگی از دروس ترمی ارائه‌شده را به دانشجویان بدهد.

\lr{-R8-3}
کندو باید امکان اضافه کردن اعلان جدید و ویرایش یا حذف اعلان‌های موجود را به معاون سطح یک (ادمین سیستم) بدهد.

\lr{-R10-1}
 کندو باید برای حساب کاربری در سطح دسترسی مدیر سطح یک یا ادمین سیستم اجازه‌ی هرگونه ویرایش و ایجاد یا حذف حساب‌های کاربری، ویرایش و حذف اطلاعات سامانه، ویرایش تنظیمات سیستم و راه‌اندازی و خاموش کردن سیستم را بدهد.
کردن سیستم را بدهد.

\lr{-R10-2}
کندو باید برای حساب کاربری در سطح دسترسی معاون سطح دو یا معاون آموزشی، اجازه‌ی ویرایش حساب‌های کاربری خود و اساتید و دانشجویان، مشاهده‌ی برنامه‌ی هفتگی چیده‌شده توسط دانشجویان منوط به اجازه‌ی آنها، ویرایش و حذف اطلاعات موجود در سامانه در سطح دسترسی تعریف شده برای معاونین آموزشی، دریافت گزارش‌ برنامه‌ها و ثبت‌نام مقدماتی، دریافت تحلیل بر روی گزارشات (شامل گزارش برنامه‌های هفتگی اخذ شده توسط دانشجویان و گزارش ارزشیابی اساتید و نظرات دانشجویان در مورد اساتید) و در نهایت بارگذاری لیست دروس ترمی را بدهد.

\lr{-R10-3}
کندو باید برای حساب‌های کاربری در سطح دسترسی دانشجو، اجازه‌ی ویرایش حساب‌ کاربری شخصی، نظردهی در مورد اساتید، ایجاد و ویرایش برنامه‌های هفتگی و انجام ثبت‌نام مقدماتی را بدهد.

\lr{-R10-4}
کندو باید برای حساب‌های کاربری در سطح دسترسی استاد، اجازه‌ی ویرایش حساب‌ کاربری شخصی، بارگذاری اطلاعات جدید (شامل جزوات، اسلایدها، نحوه‌ی ارزیابی و...) از دروس قابل‌ارائه توسط استاد و امکان ویرایش یا حذف آن‌ها  در صفحه ی شخصی را به اساتید بدهد.

\lr{-R10-6}
کندو باید به تمامی کاربران (از جمله کاربران مهمان) اجازه‌ی ورود به سیستم و خروج از سیستم را بدهد.

\lr{-R12-2}
کندو باید امکان ارسال درخواست را به دانشجویان بدهد. این درخواست‌ها برای مواردی مانند اخذ یک درس خارج از چارت، درخواست برای هم‌نیاز کردن دروس پیش‌نیاز و... است. پاسخ‌گویی به این درخواست‌ها بر عهده‌ی معاون آموزشی می‌باشد و او موظف است پس از بررسی درخواست  آن را رد یا قبول کند.

\lr{-R12-3}
کندو باید امکان طرح پرسش را به همه‌ی کاربران بدهد. کاربر پس از طرح پرسش باید مخاطب خود را نیز انتخاب کند تا پرسش مذکور برای او ارسال شود. مخاطبین، معاونین سطح یک و دو و اساتید هستند که می‌توانند در بخش پرسش‌ها آخرین سوالاتی که از آن‌ها پرسیده شده را مشاهده کنند و پاسخ دهند.

\lr{-R15-1}
تمامی معاونین می‌بایست امکان افزودن پرسش جدید در بخش پرسش‌های متداول را داشته باشند.

\lr{-R15-2}
تمامی کاربران کندو باید امکان مشاهده‌ی بخش پرسش‌های متداول را داشته باشند.

\chapter{شناسایی مورد کاربردها و مدل‌سازی تعامل کنشگر-سیستم}
در این فصل ابتدا به استنتاج مورد کاربردها
\LTRfootnote{Use Cases}
از نیازمندی‌ها پرداخته می‌شود. سپس قلمرو هر مورد کاربرد تعیین شده و زمینه‌ی موردکاربردها به تصویر کشیده می‌شود و در نهایت، تعدادی از مورد کاربردها به صورت گسترده بیان می‌شوند. هم‌چنین، تعدادی از واسط‌های کاربری سامانه‌ی کندو نیز به منظور آشنایی بیشتر با ویژگی‌های سامانه  طراحی و پیاده‌سازی شده‌اند.

\section{استنتاج مورد کاربردها از نیازمندی‌ها}
در این گام پس از اعمال مجموعه تغییراتی بر روی نیازمندی‌های کارکردی (که پیش از شروع فصل به آن‌ها اشاره شد) ، عبارت‌های فعل-اسم که بیانگر یک فرآیند‌ کسب‌وکاری هستند از این نیازمندی‌ها استخراج شده و به‌عنوان مورد کاربردها شناسایی شدند. این موردکاربردها به صورت تیتروار در بخش «تعیین قلمرو مورد کاربردها» بیان شده‌اند.

\subsection{ ساخت ماتریس ردیابی نیازمندی-موردکاربرد}
در این گام به این واقعیت که هر مورد کاربرد از یک نیازمندی به‌دست آمده است پرداخته می‌شود. این کار به کمک ماتریسی با عنوان ماتریس ردیابی نیازمندی-موردکاربرد
\LTRfootnote{Requirements Use Case Traceability Matrix (RUTM)}
صورت می‌گیرد. این ماتریس به طور دقیق مشخص می‌کند که هر مورد کاربرد از کدام نیازمندی استنتاج شده است. علاوه بر آن، در این ماتریس اولویت هر موردکاربرد طبق اولویت نیازمندی مرتبط با آن تعیین می‌شود. لازم به ذکر است که اولویت نیازمندی‌ها با توجه به هزینه‌ی مدنظر مشتری برای آن نیازمندی تعیین شده است.
ماتریس نیازمندی-موردکاربرد در 
\hyperlink{one}{شکل 4.1}
آمده است.
\newgeometry{bottom=2cm,top=1cm}
\hypertarget{one}{
\begin{figure}[!h]
\includegraphics[angle=90,width=\linewidth]{Traceability-matrix.jpg}
\caption{ماتریس ردیابی نیازمندی-موردکاربرد}
\end{figure} }
\clearpage
\restoregeometry
\subsection{تعیین قلمرو مورد کاربردها}
پس از استخراج مورد کاربردها از نیازمندی‌ها و رسم ماتریس ردیابی‌پذیری، اکنون نیاز است ﻣﻮردﮐﺎرﺑﺮدهای ﺳﻄﺢﺑﺎﻻ تعیین شوند. توسعه‌ی سیستم یک مسئله‌ی بدرفتار است و قانون توقفی برای آن وجود ندارد، یک راه‌حل مناسب برای برطرف نمودن این مشکل استفاده از مورد کاربردهای سطح بالا است. به این معنا که قلمرو هر ﻣﻮردﮐﺎرﺑﺮد باید ﻣﺸﺨﺺ شود؛ یعنی مورد کاربرد مربوطه کِی  و کجا شروع شده (
\lr{TUCBW}
\LTRfootnote{This Use Case Begins With}
)
 و کِی  تمام می‌شود (
\lr{TUCEW}
\LTRfootnote{This Use Case Ends With}
) .

 مورد کاربردهای سطح بالا عبارت‌اند از:
 
\textbf{\lr{(UC1}}
 انجام ثبت‌نام مقدماتی (کنشگر: دانشجو، سیستم: کندو)
\begin{addmargin}[.84cm]{0cm}\lr{TUCBW}
 دانشجو از منوی اصلی بر روی گزینه‌ی «ثبت‌نام مقدماتی» کلیک می‌کند.\\
\lr{TUCEW}
دانشجو به فراخور انتخاب خود پیغام‌ «تغییرات با موفقیت ثبت شد» یا «اعمال تغییرات لغو شد» را مشاهده می‌کند.

\end{addmargin}

\textbf{\lr{(UC2}}
دریافت گزارش ثبت‌نام مقدماتی (کنشگر: معاون آموزشی، سیستم: کندو)
\begin{addmargin}[.84cm]{0cm}
\lr{TUCBW}
معاون آموزشی از منوی اصلی و زیرمنوی «گزارشات» بر روی گزینه‌ی «ثبت‌نام مقدماتی» کلیک می‌کند.\\
\lr{TUCEW}
معاون آموزشی گزارش مربوط به ثبت‌نام مقدماتی را مشاهده می‌کند.
\end{addmargin}

\textbf{\lr{(UC3}}
بارگذاری لیست دروس ترمی ارائه‌شده (کنشگر: معاون آموزشی، سیستم: کندو)
\begin{addmargin}[.84cm]{0cm}\lr{TUCBW}
معاون آموزشی از صفحه‌ی «دروس ترمی» بر روی «بارگذاری دروس ترمی» کلیک می‌کند.\\
\lr{TUCEW}
معاون آموزشی پیام «دروس ترمی با موفقیت بارگذاری و به‌روزرسانی شد.» را مشاهده می‌کند.
\end{addmargin}

\textbf{\lr{(UC4}}
مشاهده‌ی لیست دروس ترمی ارائه‌شده (کنشگر: کاربر کندو، سیستم: کندو)
\begin{addmargin}[.84cm]{0cm}\lr{TUCBW}
کاربر از منوی اصلی و زیرمنوی «دروس ترمی» بر روی گزینه‌ی «ارائه‌‌شده»کلیک می‌کند.\\
\lr{TUCEW}
کاربر دروس ترمی ارائه‌‌شده را مشاهده می‌کند.
\end{addmargin}

\textbf{\lr{(UC5}}
مشاهده‌ی لیست دروس پاس‌نشده (کنشگر: دانشجو، سیستم: کندو)
\begin{addmargin}[.84cm]{0cm}\lr{TUCBW}
دانشجو از منوی اصلی و زیرمنوی «دروس ترمی» بر روی گزینه‌ی «پاس نشده» کلیک می‌کند.\\
\lr{TUCEW}
دانشجو دروس پاس‌نشده را مشاهده می‌کند.
\end{addmargin}

\textbf{\lr{(UC6}}
مشاهده‌ی دروس اولویت‌بندی‌شده (کنشگر: دانشجو، سیستم: کندو)
\begin{addmargin}[.84cm]{0cm}\lr{TUCBW}
دانشجو از منوی اصلی و زیرمنوی «دروس ترمی» بر روی گزینه‌ی «اولویت‌بندی‌‌شده» کلیک می‌کند.\\
\lr{TUCEW}
دانشجو دروس ترمی اولویت‌بندی‌‌شده را مشاهده می‌کند.
\end{addmargin}

\textbf{\lr{(UC7}}
مشاهده‌ی لیست دروس ارائه‌نشده (کنشگر: کاربر کندو، سیستم: کندو)
\begin{addmargin}[.84cm]{0cm}\lr{TUCBW}
کاربر از منوی اصلی و زیرمنوی «دروس ترمی» بر روی گزینه‌ی «ارائه نشده» کلیک می‌کند.\\
\lr{TUCEW}
کاربر دروس ترمی ارائه‌نشده را مشاهده می‌کند.
\end{addmargin}

\textbf{\lr{(UC8}}
مشاهده‌ی برنامه‌ی هفتگی پیشنهادی (کنشگر: دانشجو، سیستم: کندو)
\begin{addmargin}[.84cm]{0cm}\lr{TUCBW}
دانشجو از صفحه‌ی «برنامه‌های هفتگی» بر روی دکمه‌ی «برنامه‌ی پیشنهادی کندو» کلیک می‌کند.\\
\lr{TUCEW}
دانشجو برنامه‌ی هفتگی پیشنهادی را مشاهده می‌کند.
\end{addmargin}

\textbf{\lr{(UC9}}
 ایجاد برنامه‌های هفتگی (کنشگر: دانشجو، سیستم: کندو)
 \begin{addmargin}[.84cm]{0cm}\lr{TUCBW}
دانشجو از صفحه‌ی «برنامه‌های هفتگی» بر روی دکمه‌ی «برنامه‌ هفتگی جدید» کلیک می‌کند.\\
\lr{TUCEW}
دانشجو پیام «برنامه هفتگی با موفقیت افزوده شد.» را مشاهده می‌کند.
\end{addmargin}


\textbf{\lr{(UC10}}
ویرایش برنامه‌های هفتگی (کنشگر: دانشجو، سیستم: کندو)
 \begin{addmargin}[.84cm]{0cm}\lr{TUCBW}
دانشجو از صفحه‌ی «برنامه‌های هفتگی»  یکی از برنامه‌های هفتگی از پیش ایجاد شده‌ی خود را انتخاب می‌کند.\\
\lr{TUCEW}
دانشجو درصورت وجود مشکل در برنامه‌ی ویرایش‌شده، اخطار‌های لازم را به همراه پیغام «برنامه‌ی شما با موفقیت ذخیره شد» و یا  پیغام «اعمال تغییرات لغو شد» مشاهده می‌کند.
\end{addmargin}

\textbf{\lr{(UC11}}
مشاهده‌ی مقایسه بین برنامه‌های هفتگی (کنشگر: دانشجو، سیستم: کندو)
\begin{addmargin}[.84cm]{0cm}\lr{TUCBW}
دانشجو از صفحه‌ی «برنامه‌های هفتگی» بر روی دکمه‌ی «مقایسه‌ی برنامه‌ها» کلیک می‌کند.\\
\lr{TUCEW}
دانشجو گزارش مقایسه‌ی دو برنامه‌ی موردنظر را مشاهده می‌کند.
\end{addmargin}

\textbf{\lr{(UC12}}
مشاهده‌ی تعداد افراد ثبت‌نام‌کرده در هر درس (کنشگر: دانشجو، سیستم: کندو)
 \begin{addmargin}[.84cm]{0cm}\lr{TUCBW}
دانشجو از صفحه‌ی «ویرایش برنامه درسی» بر روی نماد آمار کلاس (که در کنار نام درس است) کلیک می‌کند.\\
\lr{TUCEW}
دانشجو تعداد افراد ثبت‌نام‌کرده در درس ‌‌مورد‌نظر را مشاهده می‌کند.
\end{addmargin}
\clearpage
\textbf{\lr{(UC13}}
دریافت گزارش برنامه‌های هفتگی (کنشگر: معاون آموزشی، سیستم: کندو)
 \begin{addmargin}[.84cm]{0cm}\lr{TUCBW}
معاون آموزشی از منوی اصلی و زیرمنوی «گزارشات» بر روی گزینه‌ی «برنامه‌های هفتگی» کلیک می‌کند.\\
\lr{TUCEW}
معاون آموزشی گزارشی از برنامه‌های هفتگی را مشاهده می‌کند.
\end{addmargin}


\textbf{\lr{(UC14}}
مشاهده‌ی تحلیل بر روی گزارش‌ها (کنشگر: معاون آموزشی، سیستم: کندو)
 \begin{addmargin}[.84cm]{0cm}\lr{TUCBW}
معاون آموزشی از منوی اصلی و منوی «گزارشات» بر روی زیرمنوی «تحلیل گزارشات» کلیک می‌کند.\\
\lr{TUCEW}
معاون آموزشی تحلیل گزارشات را مشاهده می‌کند.
\end{addmargin}

\textbf{\lr{(UC15}}
مشاهده‌ی اعلان‌ها (اطلاعیه‌ها و اخبار) (کنشگر: کاربر کندو، سیستم: کندو)
\begin{addmargin}[.84cm]{0cm}\lr{TUCBW}
کاربر از منوی اصلی بر روی گزینه‌ی «اطلاعیه‌ها و اعلان‌ها» کلیک می‌کند.\\
\lr{TUCEW}
کاربر اطلاعیه‌ها و اخبار را مشاهده می‌کند.
\end{addmargin}

\textbf{\lr{(UC16}}
اضافه کردن اعلان جدید (کنشگر: ادمین سیستم، سیستم: کندو)
\begin{addmargin}[.84cm]{0cm}\lr{TUCBW}
ادمین از بخش «مدیریت اعلان‌ها» بر روی دکمه‌ی «اعلان جدید» کلیک می‌کند.\\
\lr{TUCEW}
ادمین پیام «اعلان با موفقیت افزوده شد.» را مشاهده می‌کند.
\end{addmargin}

\textbf{\lr{(UC17}}
ویرایش و حذف اعلان‌های موجود (کنشگر: ادمین سیستم، سیستم: کندو)
\begin{addmargin}[.84cm]{0cm}\lr{TUCBW}
ادمین از بخش «مدیریت اعلان‌ها» بر روی علامت حذف/ویرایش کنار اعلان کلیک می‌کند.\\
\lr{TUCEW}
ادمین پیام «اعلان با موفقیت ویرایش/حذف شد.» را مشاهده می‌کند.
\end{addmargin}

\textbf{\lr{(UC18}}
مشاهده‌ی اطلاعات دانشجو (کنشگر: دانشجو، سیستم: کندو)\begin{addmargin}[.84cm]{0cm}\lr{TUCBW}
دانشجو بر روی گزینه‌ی «اطلاعات دانشجو» از منوی اصلی کلیک می‌کند.\\
\lr{TUCEW}
دانشجو اطلاعات کامل مربوط به خودش را مشاهده می‌کند.
\end{addmargin}

\textbf{\lr{(UC19}}
مشاهده‌ی اطلاعات اساتید (کنشگر: کاربر کندو، سیستم: کندو)\begin{addmargin}[.84cm]{0cm}\lr{TUCBW}
کاربر از منوی اصلی بر روی گزینه‌ی «اطلاعات جامع اساتید» کلیک می‌کند.\\
\lr{TUCEW}
کاربر اطلاعات مربوط به اساتید را مشاهده می‌کند.
\end{addmargin}
\clearpage
\textbf{\lr{(UC20}}
مشاهده‌ی اطلاعات دروس (کنشگر: کاربر کندو، سیستم: کندو)\begin{addmargin}[.84cm]{0cm}\lr{TUCBW}
کاربر در بخش دروس ترمی بر روی اطلاعات درس کنار نام درس کلیک می‌کند.\\
\lr{TUCEW}
کاربر اطلاعات مربوط به درس را مشاهده می‌کند.
\end{addmargin}

\textbf{\lr{(UC21}}
ایجاد حساب‌های کاربری (کنشگر: ادمین سیستم، سیستم: کندو)\begin{addmargin}[.84cm]{0cm}\lr{TUCBW}
ادمین از بخش «مدیریت کاربران» بر روی دکمه‌ی «کاربر جدید» کلیک می‌کند.\\
\lr{TUCEW}
ادمین پیام «کاربر با موفقیت افزوده شد.» را مشاهده می‌کند.
\end{addmargin}

\textbf{\lr{(UC22}}
ویرایش حساب‌های کاربری (کنشگر: ادمین سیستم، سیستم: کندو)
\begin{addmargin}[.84cm]{0cm}\lr{TUCBW}
ادمین از بخش «مدیریت کاربران» بر روی علامت ویرایش کنار نام کاربر کلیک می‌کند.\\
\lr{TUCEW}
ادمین پیام «اطلاعات کاربر با موفقیت به‌روزرسانی شد.» را مشاهده می‌کند.
\end{addmargin}

\textbf{\lr{(UC23}}
حذف حساب‌های کاربری (کنشگر: ادمین سیستم، سیستم: کندو)
\begin{addmargin}[.84cm]{0cm}\lr{TUCBW}
ادمین از بخش «مدیریت کاربران» بر روی علامت حذف کنار نام کاربر کلیک می‌کند.\\
\lr{TUCEW}
ادمین پیام « کاربر با موفقیت حذف شد.» را مشاهده می‌کند.
\end{addmargin}

\textbf{\lr{(UC24}}
ویرایش و حذف اطلاعات موجود (کنشگر: ادمین سیستم، سیستم: کندو)
\begin{addmargin}[.84cm]{0cm}\lr{TUCBW}
ادمین از منوی اصلی و زیرمنوی «تنظیمات» بر روی گزینه‌ی «ویرایش و حذف اطلاعات موجود» کلیک می‌کند.\\
\lr{TUCEW}
ادمین پیام «اطلاعات با موفقیت به‌روزرسانی شد.» را مشاهده می‌کند.
\end{addmargin}

\textbf{\lr{(UC25}}
ویرایش تنظیمات سیستم (کنشگر: ادمین سیستم، سیستم: کندو)\begin{addmargin}[.84cm]{0cm}\lr{TUCBW}
ادمین از منوی اصلی بر روی گزینه‌ی «تنظیمات» کلیک می‌کند.\\
\lr{TUCEW}
ادمین پیام «تنظیمات سیستم با موفقیت به‌روزرسانی شد.« را مشاهده می‌کند.
\end{addmargin}

\textbf{\lr{(UC26}}
راه‌اندازی سیستم (کنشگر: ادمین سیستم، سیستم: کندو)
\begin{addmargin}[.84cm]{0cm}\lr{TUCBW}
ادمین از بخش «تنظیمات» بر روی دکمه‌ی «فعالسازی سیستم» کلیک می‌کند.\\
\lr{TUCEW}
ادمین پیام «سیستم فعال شد و از‌این‌پس برای کاربران دانشگاه قابل‌دسترسی می باشد.» را مشاهده می‌کند.
\end{addmargin}

\textbf{\lr{(UC27}}
خاموش کردن سیستم (کنشگر: ادمین سیستم، سیستم: کندو)\begin{addmargin}[.84cm]{0cm}\lr{TUCBW}
ادمین از بخش «تنظیمات» بر روی دکمه‌ی «غیرفعالسازی سیستم» کلیک می‌کند.\\
\lr{TUCEW}
ادمین پیام «سیستم غیرفعال شد و از‌این‌پس برای کاربران دانشگاه قابل‌دسترسی نمی باشد.» را مشاهده می‌کند.
\end{addmargin}

\textbf{\lr{(UC28}}
ویرایش حساب کاربری شخصی (کنشگر: معاون آموزشی، سیستم: کندو)\begin{addmargin}[.84cm]{0cm}\lr{TUCBW}
معاون آموزشی از بخش «پروفایل شخصی» بر روی دکمه‌ی ویرایش کلیک می‌کند.\\
\lr{TUCEW}
معاون آموزشی پیام «پروفایل شما با موفقیت به‌روزرسانی شد.» را مشاهده می‌کند.
\end{addmargin}

\textbf{\lr{(UC29}}
ویرایش حساب کاربری اساتید (کنشگر: معاون آموزشی، سیستم: کندو)\begin{addmargin}[.84cm]{0cm}\lr{TUCBW}
معاون آموزشی از بخش «پروفایل استاد» بر روی دکمه‌ی «ویرایش» کلیک می‌کند.\\
\lr{TUCEW}
معاون آموزشی پیام «پروفایل استاد با موفقیت به‌روزرسانی شد.» را مشاهده می‌کند.
\end{addmargin}

\textbf{\lr{(UC30}}
ویرایش حساب کاربری دانشجویان (کنشگر: معاون آموزشی، سیستم: کندو)
\begin{addmargin}[.84cm]{0cm}\lr{TUCBW}
معاون آموزشی از بخش «پروفایل دانشجو» بر روی دکمه‌ی ویرایش کلیک می‌کند.\\
\lr{TUCEW}
معاون آموزشی پیام «پروفایل دانشجو با موفقیت به‌روزرسانی شد.» را مشاهده می‌کند.
\end{addmargin}

\textbf{\lr{(UC31}}
مشاهده‌ی برنامه‌های هفتگی ایجادشده توسط دانشجویان منوط به اجازه‌ی آن‌ها(کنشگر: معاون آموزشی، سیستم: کندو)
\begin{addmargin}[.84cm]{0cm}\lr{TUCBW}
معاون آموزشی از بخش «برنامه‌های هفتگی» بر روی نام دانشجوی موردنظر کلیک می‌کند.\\
\lr{TUCEW}
معاون آموزشی برنامه هفتگی مربوط به دانشجو را مشاهده می‌کند.
\end{addmargin}

\textbf{\lr{(UC32}}
ویرایش و حذف اطلاعات موجود در سطح دسترسی تعریف‌شده(کنشگر: معاون آموزشی، سیستم: کندو)
\begin{addmargin}[.84cm]{0cm}\lr{TUCBW}
معاون آموزشی از منوی اصلی و زیرمنوی «تنظیمات» بر روی گزینه‌ی «ویرایش و حذف اطلاعات موجود» کلیک می‌کند.\\
\lr{TUCEW}
معاون آموزشی پیام «اطلاعات با موفقیت به‌روزرسانی شد.» را مشاهده می‌کند.
\end{addmargin}

\textbf{\lr{(UC33}}
ویرایش حساب کاربری شخصی (کنشگر: دانشجو، سیستم: کندو)
\begin{addmargin}[.84cm]{0cm}\lr{TUCBW}
دانشجو از بخش «پروفایل شخصی» بر روی دکمه‌ی ویرایش کلیک می‌کند.\\
\lr{TUCEW}
دانشجو پیام «پروفایل شما با موفقیت به‌روزرسانی شد.» را مشاهده می‌کند.
\end{addmargin}

\textbf{\lr{(UC34}}
نظردهی درمورد اساتید (کنشگر: دانشجو، سیستم: کندو)
\begin{addmargin}[.84cm]{0cm}\lr{TUCBW}
دانشجو از بخش «تحلیل عملکرد استاد» بر روی «افزودن نظر جدید» کلیک می‌کند.\\
\lr{TUCEW}
دانشجو پیام «نظر شما با موفقیت ذخیره شد.» را مشاهده می‌کند.
\end{addmargin}

\textbf{\lr{(UC35}}
ویرایش حساب کاربری شخصی (کنشگر: استاد، سیستم: کندو)
\begin{addmargin}[.84cm]{0cm}\lr{TUCBW}
استاد از بخش «پروفایل شخصی» بر روی دکمه‌ی ویرایش کلیک می‌کند.\\
\lr{TUCEW}
استاد پیام «پروفایل شما با موفقیت به‌روزرسانی شد.» را مشاهده می‌کند.
\end{addmargin}

\textbf{\lr{(UC36}}
بارگذاری اطلاعات جدید از دروس قابل‌ارائه در صفحه‌ی شخصی استاد (کنشگر: استاد، سیستم: کندو)\begin{addmargin}[.84cm]{0cm}\lr{TUCBW}
استاد از بخش «دروس من» بر روی دکمه‌ی «بارگذاری اطلاعات» کلیک می‌کند.\\
\lr{TUCEW}
استاد پیام «اطلاعات دروس ارائه‌‌شده با موفقیت به‌روزرسانی شد.» را مشاهده می‌کند.
\end{addmargin}

\textbf{\lr{(UC37}}
مشاهده‌ی برنامه‌ی همراهان (کنشگر: دانشجو، سیستم: کندو)
\begin{addmargin}[.84cm]{0cm}\lr{TUCBW}
دانشجو از صفحه‌ی پروفایل همراه خود بر روی دکمه‌ی «برنامه‌های هفتگی» کلیک می‌کند.\\
\lr{TUCEW}
دانشجو برنامه‌های هفتگی به اشتراک گذاشته‌شده همراه خود را مشاهده می‌کند.
\end{addmargin}

\textbf{\lr{(UC38}}
ورود به سیستم (کنشگر: کاربر کندو، سیستم: کندو)\begin{addmargin}[.84cm]{0cm}\lr{TUCBW}
کاربر آدرس سایت را در مرورگر وارد کرده و دکمه‌ی Enter را فشار می دهد. \\
\lr{TUCEW}
کاربر صفحه‌ی کاربری خود را مشاهده می‌کند.
\end{addmargin}

\textbf{\lr{(UC39}}
خروج از سیستم (کنشگر: کاربر کندو، سیستم: کندو)\begin{addmargin}[.84cm]{0cm}\lr{TUCBW}
کاربر از پایین منوی اصلی بر روی «خروج از سامانه» کلیک می‌کند.\\
\lr{TUCEW}
کاربر پیام «شما با موفقیت خارج شدید.» را مشاهده می‌کند.
\end{addmargin}

\textbf{\lr{(UC40}}
اضافه‌کردن همراهان (کنشگر: معاون آموزشی، سیستم: کندو)\begin{addmargin}[.84cm]{0cm}\lr{TUCBW}
معاون آموزشی روی گزینه‌ی «همراهان» از منوی اصلی کلیک می‌کند.\\
\lr{TUCEW}
معاون آموزشی پیام «کاربر با موفقیت به همراهان شما اضافه گردید.» را مشاهده می‌کند.
\end{addmargin}

\textbf{\lr{(UC41}}
اضافه‌کردن همراهان (کنشگر: استاد، سیستم: کندو)
\begin{addmargin}[.84cm]{0cm}\lr{TUCBW}
استاد روی گزینه‌ی «همراهان» از منوی اصلی کلیک می‌کند.\\
\lr{TUCEW}
استاد پیام «کاربر با موفقیت به همراهان شما اضافه گردید.» را مشاهده می‌کند.
\end{addmargin}

\textbf{\lr{(UC42}}
اضافه‌کردن همراهان (کنشگر: دانشجو، سیستم: کندو)
\begin{addmargin}[.84cm]{0cm}\lr{TUCBW}
دانشجو روی گزینه‌ی «همراهان» از منوی اصلی کلیک می‌کند.\\
\lr{TUCEW}
دانشجو پیام «کاربر با موفقیت به همراهان شما اضافه گردید.» را مشاهده می‌کند.
\end{addmargin}

\textbf{\lr{(UC43}}
چت‌کردن با همراهان (کنشگر: معاون آموزشی، سیستم: کندو)
\begin{addmargin}[.84cm]{0cm}\lr{TUCBW}
معاون آموزشی از بخش «چت» بر روی نام همراه کلیک می‌کند.\\
\lr{TUCEW}
معاون آموزشی پیام «چت خاتمه یافت.» را مشاهده می‌کند.
\end{addmargin}

\textbf{\lr{(UC44}}
چت‌کردن با همراهان (کنشگر: دانشجو، سیستم: کندو)\begin{addmargin}[.84cm]{0cm}\lr{TUCBW}
دانشجو از بخش «چت» بر روی نام همراه کلیک می‌کند.\\
\lr{TUCEW}
دانشجو پیام «چت خاتمه یافت.» را مشاهده می‌کند.
\end{addmargin}

\textbf{\lr{(UC45}}
چت‌کردن با همراهان (کنشگر: استاد، سیستم: کندو)
\begin{addmargin}[.84cm]{0cm}\lr{TUCBW}
استاد از بخش «چت» بر روی نام همراه کلیک می‌کند.\\
\lr{TUCEW}
استاد پیام «چت خاتمه یافت.» را مشاهده می‌کند.
\end{addmargin}

\textbf{\lr{(UC46}}
ارسال درخواست (کنشگر: دانشجو، سیستم: کندو)
\begin{addmargin}[.84cm]{0cm}\lr{TUCBW}
دانشجو در صفحه‌ی «ویرایش برنامه هفتگی» بر روی گزینه‌ی ارسال درخواست مجوز کنار پیام خطا کلیک می‌کند.\\
\lr{TUCEW}
دانشجو پیام «درخواست شما با موفقیت ثبت شد.» ر ا مشاهده می‌کند.
\end{addmargin}

\textbf{\lr{(UC47}}
طرح پرسش (کنشگر: کاربر کندو، سیستم: کندو)
\begin{addmargin}[.84cm]{0cm}\lr{TUCBW}
کاربر از بخش «تیکت‌ها» بر روی دکمه‌ی «تیکت جدید» کلیک می‌کند.\\
\lr{TUCEW}
کاربر پیام «تیکت شما با موفقیت ثبت شد.» را مشاهده می‌کند.
\end{addmargin}

\textbf{\lr{(UC48}}
بررسی و پاسخ به پرسش‌ها (کنشگر: ادمین سیستم، سیستم: کندو)
\begin{addmargin}[.84cm]{0cm}\lr{TUCBW}
ادمین از بخش «مدیریت تیکت‌ها» بر روی دکمه‌ی «ثبت پاسخ» کنار تیکت مربوطه کلیک می‌کند.\\
\lr{TUCEW}
ادمین پیام «پاسخ شما با موفقیت ثبت شد.» را مشاهده می‌کند.
\end{addmargin}

\textbf{\lr{(UC49}}
بررسی و پاسخ به پرسش‌ها (کنشگر: معاون آموزشی، سیستم: کندو)
\begin{addmargin}[.84cm]{0cm}\lr{TUCBW}
معاون آموزشی از بخش «مدیریت تیکت‌ها» بر روی دکمه‌ی «ثبت پاسخ» کنار تیکت مربوطه کلیک می‌کند.\\
\lr{TUCEW}
معاون آموزشی پیام «پاسخ شما با موفقیت ثبت شد.» را مشاهده می‌کند.
\end{addmargin}

\textbf{\lr{(UC50}}
بررسی و پاسخ به پرسش‌ها (کنشگر: استاد، سیستم: کندو)
\begin{addmargin}[.84cm]{0cm}\lr{TUCBW}
استاد از بخش «مدیریت تیکت‌ها» بر روی دکمه‌ی «ثبت پاسخ» کنار تیکت مربوطه کلیک می‌کند.\\
\lr{TUCEW}
استاد پیام «پاسخ شما با موفقیت ثبت شد.» را مشاهده می‌کند.
\end{addmargin}

\textbf{\lr{(UC51}}
بررسی و پاسخ به درخواست‌ها (کنشگر: معاون آموزشی، سیستم: کندو)
\begin{addmargin}[.84cm]{0cm}\lr{TUCBW}
معاون آموزشی از بخش«مدیریت درخواست‌ها» بر روی دکمه‌ی تایید/رد درخواست کنار پرسش مربوطه کلیک می‌کند.\\
\lr{TUCEW}
معاون آموزشی به فراخور انتخابی که در مرحله قبل داشت یکی از پیام‌های «تغییرات با موفقیت ثبت شد.» و «اعمال تغییرات متوقف شد.» را مشاهده می‌کند.
\end{addmargin}

\textbf{\lr{(UC52}}
بارگذاری اطلاعات جدید در سامانه به شکل دستی و خودکار (کنشگر: ادمین سیستم، سیستم: کندو)
\begin{addmargin}[.84cm]{0cm}\lr{TUCBW}
ادمین در بخش تنظیمات بر روی «بارگذاری اطلاعات» کلیک می‌کند.\\
\lr{TUCEW}
ادمین پیام «اطلاعات با موفقیت بارگذاری شد.» را مشاهده می‌کند.
\end{addmargin}

\textbf{\lr{(UC53}}
بارگذاری اطلاعات جدید در سامانه به شکل دستی و خودکار (کنشگر: معاون آموزشی، سیستم: کندو)
\begin{addmargin}[.84cm]{0cm}\lr{TUCBW}
معاون آموزشی در بخش تنظیمات بر روی «بارگذاری اطلاعات» کلیک می‌کند.\\
\lr{TUCEW}
معاون آموزشی پیام «اطلاعات با موفقیت بارگذاری شد.» را مشاهده می‌کند.
\end{addmargin}

\textbf{\lr{(UC54}}
بارگیری فایل‌های موجود در سطح دسترسی تعریف‌شده (کنشگر: کاربر کندو، سیستم: کندو)
\begin{addmargin}[.84cm]{0cm}\lr{TUCBW}
کاربر بر روی گزینه‌ی دانلود فایل کلیک می‌کند.\\
\lr{TUCEW}
کاربر پیام «فایل ‌‌مورد‌نظر با موفقیت دانلود شد.» را مشاهده می‌کند.
\end{addmargin}

\textbf{\lr{(UC55}}
اضافه‌کردن پرسش جدید به بخش پرسش‌های متداول (کنشگر: معاون آموزشی، سیستم: کندو)
\begin{addmargin}[.84cm]{0cm}\lr{TUCBW}
معاون آموزشی از بخش «مدیریت پرسش‌های متداول» بر روی دکمه‌ی «پرسش متداول جدید» کلیک می‌کند.\\
\lr{TUCEW}
معاون آموزشی پیام «پرسش متداول جدید با موفقیت ثبت شد.« را مشاهده می‌کند.
\end{addmargin}

\textbf{\lr{(UC56}}
مشاهده‌ی بخش پرسش‌های متداول (کنشگر: کاربر‌ کندو، سیستم: کندو)
\begin{addmargin}[.84cm]{0cm}\lr{TUCBW}
کاربر بر روی «پرسش‌های متداول» از منوی اصلی کلیک می‌کند.\\
\lr{TUCEW}
کاربر پرسش‌های متداول را مشاهده می‌کند.
\end{addmargin}
\clearpage
\subsection{مصورسازی زمینه‌ی مورد کاربردها}
پس از تعیین قلمرو هر مورد کاربرد، در این گام نمودار مورد کاربردها که نمایش‌دهنده‌ی کنشگرها و روابط ارث‌بری بین آن‌ها، موردکاربردها و روابط بین آن‌ها و مرز سیستم‌ها و زیرسیستم‌ها است ترسیم می‌شود. هدف از این کار سازمان‌دهی و مصورسازی اطلاعات است. این نمودار ابتدا مبتنی بر گونه افرازبندی شد، اما به دلیل تعداد نسبتاً زیاد مورد کاربردها و پیچیدگی و غیرقابل‌فهم بودن نمودار حاصل، سرانجام افراز مبتنی بر نقش به ‌عنوان افرازبندی نهایی انتخاب شد. 
\hyperlink{two}{شکل 4.2}
نمودار مورد کاربردها با افراز مبتنی بر نقش را نمایش می‌دهد.
\hypertarget{two}{
\begin{figure}[!h]
\includegraphics[width=\linewidth]{usecaseD.jpg}
\caption{نمودار مورد کاربردها}
\end{figure}}
\clearpage
\subsection{تخصیص مورد کاربردها به تکرارها}
در گام‌های پیشین، ﻣﻮردﮐﺎرﺑﺮدﻫﺎ ﺷﻨﺎﺳﺎﯾﯽ وﺑﻪ ﺗﺼﻮﯾﺮﻛﺸﯿﺪهﺷﺪند و اوﻟﻮﯾﺖ هرکدام نیز طبق اوﻟﻮﯾﺖ ﻧﯿﺎزﻣﻨﺪی‌ها ﺑﻪ دﺳﺖآﻣﺪ. هدف این گام تولید یک زمان‌بندی مناسب برای توسعه و تحویل مورد کاربردها (برنامه‌ریزی تکرارها بر اساس مورد کاربردها) است. نتیجه‌ی نهایی این برنامه‌ریزی جدول تخصیص مورد کاربردها به تکرارها می‌باشد که در دو 
 \hyperlink{three}{شکل 4.3}
و
 \hyperlink{four}{4.4}
 به نمایش درآمده است. برای تهیه‌ی این جدول ابتدا تلاش مورد نیاز برای توسعه و تحویل هر مورد کاربرد بر حسب نفر-هفته تخمین زده شده و در ستون مربوطه نوشته می‌شود. پس از مشخص کردن وابستگی بین مورد کاربردها، برای هر تکرار یک زمان‌بندی مطابق با اولویت و وابستگی مورد کاربردها و توانایی تیم توسعه تولید می‌شود.\\
همان‌گونه که در جدول نیز مشخص است، این پروژه در ۴ تکرار ۴ هفته‌ای و با یک تیم ۶ نفره به پایان می‌رسد.
\hypertarget{three}{
\begin{figure}[!h]
\includegraphics[width=\linewidth]{tekrar1.jpg}
\caption{جدول تخصیص موردکاربردها به تکرارها - بخش اول}
\end{figure}}
\clearpage
\hypertarget{four}{
\begin{figure}[!h]
\includegraphics[width=\linewidth]{tekrar2.jpg}
\caption{جدول تخصیص موردکاربردها به تکرارها - بخش دوم}
\end{figure}}
\section{مدل‌سازی تعامل کنشگر-سیستم}
پیش‌تر اشاره شد که هر مورد کاربرد (که بیانگر یک فرآیند کسب وکار است) با یک کنشگر شروع و با همان کنشگر به اتمام می‌رسد و یک وظیفه‌ی کسب وکاری را برای وی انجام می‌دهد. برای انجام این وظیفه، سیستم باید با کنشگر تعامل داشته باشد تا بتواند مورد کاربرد مربوطه را عملی کند که این تعامل،
 تعامل کنشگر-سیستم نامیده می‌شود.

\subsection{تعیین گام‌های تعامل کنشگر-سیستم}
 برای مشخص کردن جزئیات این تعامل، ابتدا شش 
  \hypertarget{ch4}{ موردکاربردی} 
 که تأثیر مهم‌تری در روند پروژه داشتند و برای مشتری و به تبع تیم توسعه از اهمیت بیشتری برخوردار بودند، از میان مورد کاربردها انتخاب شدند. سپس جدول‌های دو ستونی برای مورد کاربردهای مذکور ترسیم شدند. در ادامه، به کمک مورد کاربردهای سطح بالا که در بخش گذشته به دست آمده بودند شروع و پایان جدول‌ها تعیین شده و در نهایت، گام صفر و گام‌های تعامل کنشگر-سیستم در هر ستون‌ جدول (ضمن بیان پیش‌شرط یا پس‌شرط در صورت نیاز) مشخص شدند.
 
\subsection{استفاده از پیش‌نمونه‌های واسط کاربری}
در این گام در راستای به‌دست‌آوردن دید بهتری از محصول  نهایی و جزییات بیشتری از تعامل کنشگر با سیستم،  واسط کاربری گرافیکی سامانه‌ی کندو برای شش مورد کاربردی که در بخش قبل انتخاب شدند پیاده‌سازی شده است. این واسط‌های گرافیکی به سبب این که تعامل کنشگر-سیستم را به صورت مطلوب‌تری پیاده‌سازی می‌کنند و تصویر روشنی از نتیجه‌ی محصول نشان می‌دهند، بازخوردهایی را به همراه خواهند داشت که برای برطرف نمودن ذهنیت اشتباه و خطاهای احتمالی تیم بسیار مفید خواهد بود. هم‌چنین، از مطالب این بخش در تهیه‌ی کاتالوگ و بروشور محصول برای ذی‌نفعان می‌توان در آینده بهره برد.
 
در این قسمت، ابتدا جدول موردکاربرد گسترده‌ برای هر کدام از شش مورد کاربرد مذکور رسم شده‌ است و در ادامه‌ی هر یک، واسط‌های کاربری گرافیکی مربوط به آن‌ آورده شده است. 
\clearpage
\newgeometry{bottom=2cm,top=1.5cm}
\textbf{\lr{(UC1}}
انجام ثبت‌نام مقدماتی (کنشگر: دانشجو، سیستم:کندو)
\begin{figure}[!htb]
\centering
\includegraphics[width=\linewidth]{sabtnamm.jpg}
\caption{جدول گسترد‌ه‌ ثبت‌نام مقدماتی}
\end{figure}
\vspace{\baselineskip}
\begin{figure}[!htb]
\centering
\includegraphics[width=16cm]{sabtnn.png}
\caption{واسط کاربری ثبت‌نام مقدماتی}
\end{figure} 
\pagebreak
\clearpage
\restoregeometry
\textbf{\lr{(UC10}}
ویرایش برنامه‌های هفتگی (کنشگر: دانشجو، سیستم:کندو)
\begin{figure}[!htb]
\centering
\includegraphics[width=\linewidth]{vbar.jpg}
\caption{جدول گسترده‌ ویرایش برنامه‌‌های هفتگی}
\end{figure}
\clearpage
\begin{figure}
\centering
\includegraphics[width=17cm]{sabtnam.png}
\caption{واسط کاربری ویرایش برنامه‌‌های هفتگی}
\end{figure}
\clearpage
\newgeometry{bottom=2cm,top=1.5cm}
\lr{\textbf{(UC14}}
ﻣﺸﺎﻫﺪﻩ‌ﯼ ﺗﺤﻠﯿﻞﺑﺮﺭﻭﯼﮔﺰﺍﺭﺵ‌ﻫﺎ (کنشگر: معاون آموزشی، ﺳﯿﺴﺘﻢ:ﮐﻨﺪﻭ)
\begin{figure}[!htb]
\centering
\includegraphics[width=15cm]{tahlilg.jpg}
\caption{جدول گسترده‌ تحلیل گزارش‌ها}
\end{figure} 
\vspace{\baselineskip}
\begin{figure}[!htb]
\centering
\includegraphics[width=16cm]{ostad.png}
\caption{ واسط کاربری تحلیل گزارش‌ها - تحلیل گزارش اساتید}
\end{figure}
\pagebreak
\clearpage
\textbf{\lr{(UC42}}
اضافه کردن همراهان (کنشگر: دانشجو، سیستم:کندو)
\begin{figure}[!htb]
\centering
\includegraphics[width=\linewidth]{followers.jpg}
\caption{جدول گسترده‌ اضافه کردن همراهان}
\end{figure}
\begin{figure}[!htb]
\centering
\includegraphics[width=16cm]{followers.png}
\caption{واسط کاربری اضافه کردن همراهان}
\end{figure}
\pagebreak
\clearpage
\lr{\textbf{(UC46}}
ارسال درخواست (کنشگر: دانشجو، ﺳﯿﺴﺘﻢ:ﮐﻨﺪﻭ)
\begin{figure}[!htb]
\centering
\includegraphics[width=\linewidth]{requestSend.jpg}
\caption{جدول گسترده‌ ارسال درخواست}
\end{figure} 
\begin{figure}[!htb]
\centering
\includegraphics[width=16cm]{ui-requestSend.png}
\caption{واسط کاربری ارسال درخواست}
\end{figure} 
\pagebreak
\clearpage
\restoregeometry
\textbf{\lr{(UC51}}
بررسی و پاسخ به درخواست‌ها (کنشگر: معاون آموزشی، سیستم:کندو)
\begin{figure}[!htb]
\centering
\includegraphics[width=\linewidth]{requestAns.jpg}
\caption{جدول گسترده‌ بررسی و پاسخ به درخواست‌ها}
\end{figure}
\begin{figure}[!htb]
\centering
\includegraphics[width=17cm]{ui-requestAns.png}
\caption{واسط کاربری بررسی و پاسخ به درخواست‌ها}
\end{figure}
\pagebreak
\clearpage

\chapter{مدل‌سازی تعامل شئ}
همان‌گونه که پیش‌تر بیان شد، هر فرآیند کسب‌وکاری به طور مشترک توسط کنشگر و سیستم در دو بخش صورت می‌گیرد: بخش اول قسمت پیش‌زمینه‌ای تعامل کنشگر-سیستم است و بخش دوم قسمت پس‌زمینه‌ای پردازش سیستم. در
\hyperlink{ch4}{ بخش دوم فصل ۴}
 توضیح داده شد که چگونه با استفاده از مورد کاربردهای گسترده، تعامل کنشگر-سیستم مشخص می‌شود. در این فصل به مدل‌سازی و تبیین این موضوع پرداخته می‌شود که اشیاء چگونه با یک‌دیگر تعامل می‌کنند تا فرایندهای کسب‌وکاری را به انجام برساند.

این مدل‌سازی در گام‌های زیر پیش رفته است:\\
گام ۱) جمع آوری اطلاعات درباره‌ی فرایند کسب وکار\\
پیش از شروع کار، بار دیگر مبحث جمع‌آوری اطلاعات و مرور دامنه‌ی کاربرد توسط تیم توسعه انجام شد؛  هدف از این مرور دست‌یابی اعضای تیم به تفکری یکسان درموردمفاهیم دامنه‌ی کاربرد بود. در انتهای این گام تمامی اعضای تیم توسعه دید خوبی نسبت به دامنه کاربرد بدست آوردند و نقطه نظرات اعضای تیم در مورد دامنه‌ی کاربرد و خواسته های مشتریان بسیار به هم نزدیک شده بود.

گام ۲) نوشتن توصیف سناریوها برای گام‌های غیربدیهی از مورد کاربردهای گسترده\\
 از میان شش موردکاربردی که به صورت گسترده در بخش‌های قبل بیان شد، سه موردکاربرد مهم‌تر انتخاب شدند.
ابتدا گام غیربدیهی هر یک از این سه مورد کاربرد شناسایی و در جدول موردکاربرد گسترده‌ مربوطه علامت زده شد. سپس متن سناریوها به تحریر درآمد؛ این متن‌ها در قالب دنباله‌ای از جملات اعلانی است که هر یک چگونگی تعامل  اشیاء با یکدیگر به منظور تحقق گام غیربدیهی مذکور را تعیین می‌کند.

گام ۳) ساخت جدول‌های سناریو\\
در این گام متن هر سناریوکه در بخش قبل نوشته شده بود در جدول سناریو به نمایش در آمد. این جدول‌ها کمک کردند تا ساخت نمودار توالی در گام بعد به شکل بهینه‌تر و در زمان کمتری صورت گیرد.

گام ۴) رسم نمودارهای توالی از جدول‌های سناریو\\
در این گام نمودارهای توالی 
\lr{UML}
 بر طبق متن و جدول‌های سناریو که پیش‌تر تهیه شده بودند ترسیم  شدند. برای رسم بهتر و دقیق‌تر این نمودارها، از مدل دامنه برای تعیین برخی از کلاس‌ها و اشیاء و هم‌چنین از نمودار لایه‌ای  معماری سیستم (که در فصل سوم ترسیم شد) برای تعیین لایه‌های درگیر در هر سناریو استفاده گردید.

گام ۵) مرور و بازبینی مدل‌های تعامل شی\\
در این گام بار دیگرمدل‌های تعامل شی از نظر سازگاری،کامل بودن و درستی مورد بازبینی و بازنگری تیم توسعه قرار گرفت. این بازنگری اشکالاتی که در برخی گام‌ها و شماره‌ی آن‌ها رخ داده بود (که در متن سناریوها و به تبع جدول سناریو‌ها تأثیر داشت) را برطرف نمود.

در ادامه، به ترتیب برای هر یک از سه مورد کاربرد انتخاب شده ابتدا گام غیر بدیهی در جدول مورد کاربرد گسترده مربوط به هر کدام مشخص شده است. سپس متن سناریوی آن گام، جدول سناریو و در نهایت نمودار توالی
\lr{UML}
 مربوطه رسم شده است.
 \clearpage
\subsection*{سناریوی اول}


\textbf{\lr{(UC51}}
بررسی و پاسخ به درخواست‌ها (کنشگر: معاون آموزشی، سیستم:کندو)
\begin{figure}[!htb]
\centering
\includegraphics[width=\linewidth]{sen1.jpg}
\caption{مشخص کردن گام غیر بدیهی در جدول مورد کاربرد گسترده}
\end{figure} 
\clearpage
گام ششم یک گام غیربدیهی بوده و سناریوی زیر برای این گام نوشته شده است:

5) معاون آموزشی پس از بررسی همه و یا تعدادی از درخواست‌ها، در پایین صفحه \\
5.1) برای ذخیره‌ کردن تغییرات دکمه‌ی «اعمال تغییرات» را کلیک می‌کند.\\
5.2) برای ذخیره‌ نشدن تغییرات دکمه‌ی «انصراف» را کلیک می‌کند.\\
6.1) واسط گرافیکی پاسخ درخواست، نتیجه‌ی کلیک کاربر را برای کنترلر پاسخ درخواست ارسال می‌کند.\\
6.2)کنترلر پاسخ درخواست نتیجه‌ی کلیک را برای سرویس پاسخ درخواست ارسال می‌کند.\\
6.3)سرویس پاسخ درخواست یک پیغام خالی 
\lr{msg}
 ایجاد می‌کند.\\
6.4) اگر دکمه‌ی اعمال تغییرات کلیک شده باشد\\
6.4.1) اگر لیست درخواست خالی نباشد\\
6.4.1.1) برای هر کدام از درخواست ها\\
6.4.1.1.1) سرویس پاسخ درخواست با استفاده از شناسه‌ی درخواست، مقدار متغیر 
\lr{accepted}
 از شیء
 \lr{userRespones}
   را به مقدار متغیر
\lr{accepted}
     در شیء
\lr{request}
       تغییر می‌دهد و این شی را در مدیر پایگاه داده‌ی پاسخ درخواست به‌روزرسانی می‌کند.\\
6.4.1.1.2) سرویس‌ پاسخ درخواست با استفاده از شناسه‌ی درخواست، مقدار متغیر
\lr{status}
 از شیء
\lr{request}
  را به مقدار
\lr{checked}
   تغییر می‌دهد و این شیء را در مدیر پایگاه داده‌ی پاسخ درخواست نیز به‌روزرسانی می‌کند.\\
6.4.1.2) سرویس پیغام « تغییرات با موفقیت ثبت شد» را در 
\lr{msg}
 می‌نویسد.\\
6.4.2) در غیر این صورت\\
6.4.2.1) سرویس پاسخ‌ درخواست پیغام «درخواستی برای ثبت وجود ندارد» را در 
\lr{msg}
 می‌نویسد.\\
6.5) در غیر این صورت\\
6.5.1) سرویس پاسخ درخواست پیغام «اعمال تغییرات لغو شد» را در 
\lr{msg}
 می‌نویسد.\\
6.6) سرویس پاسخ درخواست پیغام 
\lr{msg}
 را به کنترلر پاسخ درخواست ارسال می‌کند.\\
6.7) کنترلر پاسخ درخواست پیغام 
\lr{msg}
را برای واسط گرافیکی پاسخ درخواست ارسال می‌کند.\\
6.8) واسط گرافیکی پاسخ درخواست پیغام 
\lr{msg} 
را به معاون آموزشی نشان می‌دهد.\\
\clearpage
\begin{figure}[!htb]
\includegraphics[width=\linewidth]{senj1.jpg}
\caption{جدول سناریوی اول} 
\end{figure} 
\clearpage
\newgeometry{bottom=2cm,top=1cm}
\begin{figure}[!h]
\centering
\includegraphics[width=\linewidth, height=23cm]{Scenario1.png}
\caption{نمودار توالی سناریوی اول}
\end{figure}
\clearpage
\restoregeometry
\subsection*{سناریوی دوم}


\textbf{\lr{(UC10}}
ویرایش برنامه‌های هفتگی (کنشگر: دانشجو، سیستم:کندو)
\begin{figure}[!htb]
\centering
\includegraphics[width=\linewidth]{sen2.jpg}
\caption{مشخص کردن گام غیر بدیهی در جدول مورد کاربرد گسترده}
\end{figure} 
\clearpage
گام هشتم یک گام غیربدیهی بوده و سناریوی زیر برای آن نوشته شده است:

7. دانشجو پس از انجام تغییرات بر روی برنامه‌ی هفتگی خود، در پایین صفحه :\\
7.1) برای ذخیره‌ کردن تغییرات بر روی دکمه‌ی «ثبت تغییرات» کلیک می‌کند.\\
7.2) برای ذخیره نشدن تغییرات بر روی دکمه‌ی «لغو تغییرات» کلیک می‌کند.\\
7.1)  واسط گرافیکی ویرایش برنامه‌‌ی هفتگی نتیجه‌ی کلیک دانشجو را برای کنترلر ویرایش برنامه‌ی هفتگی ارسال می‌کند.\\
8.2) کنترلر ویرایش برنامه‌ی هفتگی نتیجه‌ی کلیک را برای سرویس ویرایش برنامه‌ی هفتگی ارسال می‌کند.\\
8.3) سرویس ویرایش برنامه‌ی هفتگی یک پیغام خالی 
\lr{msg}
 ایجاد می‌کند.\\
8.4) اگر دکمه‌ی «ثبت تغییرات» کلیک شده باشد:\\
8.4.1) سرویس ویرایش برنامه‌ی هفتگی یک لیست خالی 
\lr{msg warning}
 ایجاد می‌کند.\\
8.4.2) سرویس ویرایش برنامه‌ی هفتگی لیست شناسه‌ی سکشن‌ها را در مدیر پایگاه داده‌ی برنامه‌ی هفتگی به‌روزرسانی می‌کند.\\
8.4.3) سرویس ویرایش برنامه‌ی هفتگی لیست اخطارها را از سرویس ویرایش برنامه‌ی هفتگی دریافت می‌کند.\\
8.4.4) اگر لیست اخطارها خالی نباشد\\
8.4.4.1) برای هرکدام از اخطارها \\
8.4.4.1.1 ) سرویس ویرایش برنامه‌ی هفتگی متن علت اخطار را به لیست 
\lr{msg warning}
 اضافه می‌کند.\\
8.4.5) در غیر این صورت\\
8.4.5.1) سرویس ویرایش برنامه‌ی هفتگی متن «خطایی موجود نیست» را به لیست 
\lr{msg warning}
 اضافه می‌کند.\\
8.4.6) سرویس ویرایش برنامه‌ی هفتگی لیست 
\lr{msg warning}
 را به کنترلر ویرایش برنامه‌ی هفتگی ارسال می‌کند.\\
8.4.7) کنترلر ویرایش برنامه‌ی هفتگی لیست 
\lr{msg warning}
 را به واسط گرافیکی ویرایش برنامه‌ی هفتگی ارسال می‌کند.\\
\clearpage
\begin{figure}[!h]
\includegraphics[width=\linewidth]{senj2-1.jpg}
\end{figure} 
\begin{figure}[!h]
\includegraphics[width=\linewidth]{senj2-2.jpg}
\caption{جدول سناریوی دوم} 
\end{figure}
\clearpage
\newgeometry{bottom=2cm,top=1cm}
\begin{figure}[!h]
\centering
\includegraphics[width=\linewidth, height=23cm]{Scenario2.png}
\caption{نمودار توالی سناریوی دوم}
\end{figure}
\clearpage
\restoregeometry
\subsection*{سناریوی سوم}


\textbf{\lr{(UC1}}
 انجام ثبت‌نام مقدماتی (کنشگر: دانشجو، سیستم:کندو)
\begin{figure}[!htb]
\centering
\includegraphics[width=\linewidth]{sen3.jpg}
\caption{مشخص کردن گام غیر بدیهی در جدول مورد کاربرد گسترده}
\end{figure} 
\clearpage
گام ششم یک گام غیربدیهی بوده و سناریوی زیر برای آن نوشته شده است:

5) دانشجو پس از انتخاب دروس، در پایین صفحه :\\
5.1) برای ذخیره کردن بر روی دکمه‌ی «ثبت تغییرات» کلیک می‌کند.\\
5.2) برای ذخیره نشدن تغییرات بر روی دکمه‌ی «لغو تغییرات» کلیک می‌کند.\\
6.1) واسط گرافیکی ثبت‌نام مقدماتی نتیجه‌ی کلیک دانشجو را برای کنترلر ثبت‌نام مقدماتی ارسال می‌کند.\\
6.2) کنترلر ثبت‌نام مقدماتی نتیجه‌ی کلیک را برای سرویس ثبت‌نام مقدماتی ارسال می‌کند.\\
6.3) سرویس ثبت‌نام مقدماتی یک پیغام خالی 
\lr{msg}
 ایجاد می‌کند.\\
6.4) اگر دکمه‌ی «ثبت تغییرات» کلیک شده باشد:\\
6.4.1) سرویس ثبت‌نام مقدماتی لیست اخطارها را از سرویس ثبت‌نام مقدماتی دریافت می‌کند.\\
6.4.2) اگر لیست اخطارها خالی نباشد:\\
6.4.2.1) سرویس ثبت‌نام مقدماتی یک لیست خالی 
\lr{msg warning}
 ایجاد می‌کند.\\
6.4.2.2) برای هرکدام از اخطارها:\\
6.4.2.2.1) سرویس ثبت‌نام مقدماتی متن علت خطا را به لیست 
\lr{msg warning}
 اضافه می‌کند.\\
6.4.2.3) سرویس ثبت‌نام مقدماتی لیست 
\lr{msg warning}
 را به کنترلر ثبت‌نام مقدماتی ارسال می‌کند.\\
6.4.2.4) کنترلر ثبت‌نام مقدماتی لیست 
\lr{msg warning}
 را به واسط گرافیکی ثبت‌نام مقدماتی ارسال می‌کند.\\
6.4.2.5‌) واسط گرافیکی ثبت‌نام‌ مقدماتی لیست 
\lr{msg warning}
 را به دانشجو نمایش می‌دهد.\\
6.4.2.6) سرویس ثبت‌نام مقدماتی پیغام «امکان ثبت‌ تغییرات وجود ندارد، خطاها را بررسی کنید» را در
\lr{msg}
 می نویسد.\\
6.4.3) در غیر‌این‌صورت:\\
6.4.3.1) سرویس ثبت‌نام مقدماتی لیست شناسه‌ی دروس انتخاب شده را در مدیر پایگاه داده ثبت‌نام مقدماتی ذخیره می‌کند.\\
6.4.3.2) سرویس ثبت‌نام مقدماتی پیغام «ثبت‌نام مقدماتی با موفقیت انجام شد» را در 
\lr{msg}
 می نویسد.\\
6.5) در غیر این صورت\\
6.5.1) سرویس ثبت‌نام مقدماتی پیغام «اعمال تغییرات لغو شد» را  در 
\lr{msg}
 می نویسد.\\
6.6) سرویس ثبت‌نام مقدماتی پیغام 
\lr{msg}
 را به کنترلر ثبت‌نام مقدماتی ارسال می‌کند.\\
6.7) کنترلر ثبت‌نام مقدماتی پیغام 
\lr{msg}
 را به واسط گرافیکی ثبت‌نام مقدماتی ارسال می‌کند.\\
6.8) واسط گرافیکی ثبت‌نام مقدماتی پیغام 
\lr{msg}
 را به دانشجو نمایش می‌دهد.\\
\clearpage
\begin{figure}[!h]
\includegraphics[width=\linewidth]{senj3-1.jpg}
\end{figure} 
\begin{figure}[!h]
\includegraphics[width=\linewidth]{senj3-2.jpg}
\caption{جدول سناریوی سوم} 
\end{figure}
\clearpage
\newgeometry{bottom=2cm,top=1cm}
\begin{figure}[!h]
\centering
\includegraphics[width=\linewidth, height=23cm]{Scenario3.png}
\caption{نمودار توالی سناریوی سوم}
\end{figure}
\clearpage

\chapter{استنتاج نمودار کلاس طراحی}

\section{شناسایی کلاس‌ها}
برای استنتاج کلاس‌های نمودار کلاس طراحی از سناریوهای نوشته شده در فصل قبل و همچنین از مدل دامنه ترسیم شده در فاز اول استفاده شده است. درواقع از سناریوها برای استخراج اکثر کلاس‌های بسته‌های رابط گرافیکی، کنترلر، سرویس، و مخزن استفاده شده است. همچنین از مدل دامنه برای استخراج اکثر کلاس‌های بسته‌ی مدل که در لایه‌ی پایگاه داده قرار دارد، استفاده شده است. البته برخی از کلاس‌های مدل دامنه که صرفا برای مفهوم‌سازی نتایج طوفان فکری استفاده شده اند و در پیاده‌سازی نقشی ندارند در نمودار کلاس طراحی آورده نشده اند. برای مثال کلاس
\lr{Report}
که در مدل دامنه آورده شده است در نمودار کلاس طراحی قرار داده نشده است، چراکه برای دریافت گزارش‌های لازم تنها کافی است از کلاس‌های مدیر پایگاه داده مقادیر مورد نیاز که در پایگاه داده قرار دارند استخراج شوند و توسط کلاس‌های سرویس به کنترلر و درنهایت به رابط گرافیکی ارسال شوند در نتیجه نیازی به قرار دادن کلاسی مجزا با عنوان
\lr{Report}
نبوده است.


\section{شناسایی متدها}
برای شناسایی متدها از سناریوهای نوشته شده استفاده شده است. درواقع تعدادی از متدهای موجود در نمودار کلاس طراحی از سناریوها استخراج شده اند و بقیه آن‌ها با توجه به همان ۳ سناریوی اصلی و به همان شیوه استنتاج شده اند و در نهایت در نمودار کلاس آورده شده اند.

البته برای جلوگیری از گستردگی و پیچیدگی بیش از حد نمودار کلاس از آوردن متدهای
\lr{setter}
 ،
\lr{getter}
، و همچنین سازنده‌های کلاس‌ها صرف نظر شده است و تنها به متدهای اصلی اکتفا شده است.

از آنجایی که در معماری لایه‌ای استفاده شده منطق برنامه از
\lr{Api} 
نوشته شده برای ارتباظ سرور و کلاینت جدا شده است، در لایه کنترلر اکثر متدهای استفاده شده متدهای هم‌نام با متدهای لایه‌ سرویس می‌باشند با این تفاوت که در لایه کنترلر از این متدها برای ارتباط سرور و کلاینت استفاده می‌شود و
\lr{Api}
های لازم در این لایه یعنی لایه کنترلر پیاده‌سازی می‌شوند و منطق برنامه در لایه سرویس پیاده‌سازی می‌شود.


\section{شناسایی صفت‌ها}
برای شناسایی صف‌های کلاس‌ها بیشتر از مدل دامنه استفاده شده است و درواقع اکثر صف‌های کلاس‌های بسته مدل از مدل دامنه گرفته شده اند. همچنین در برخی موارد و با توجه به سناریوهای نوشته شده برخی از این صفت‌ها نسبت به مدل دامنه تغییر کرده اند. همچنین برخی از صفات که در مدل دامنه آورده نشده بودند از سناریوها گرفته شده اند.
\section{روابط بین کلاس‌ها و شناسایی روابط}
برای نشان دادن روابطی همچون ارث‌بری، تجمیع، و انجمن از مدل دامنه کمک گرفته شده است و این روابط بیشتر در بین کلاس‌های بسته مدل که موجودیت‌های سیستم را مشخص می‌کنند، وجود دارند.
همچنین برای نشان دادن روابط
\lr{create}
 ، 
\lr{use}
، و 
\lr{call}
از نمودارهای سناریو کمک گرفته شده است. این روبط در بین اکثر کلاس‌ها وجود دارد و بیشتر بین دو کلاس از دو لایه‌ی متفاوت وجود دارند. دلیل این امر استفاده از معماری لایه‌ای می‌باشد.
\section{فهرست بررسی برای بازبینی نمودار کلاس طراحی}
در نهایت پس از شناسایی کلاس‌ها، متدها، صفت‌ها، و روابط بین کلاس‌ها یک نمودار کلاس طراحی اولیه رسم شد که پس از بررسی‌های بیشتر و ایجاد تغییرات مورد نیاز نمودار کلاس طراحی زیر رسم شد:

\begin{figure}[h!]
  \includegraphics[width=\linewidth]{FinalDCD.png}
  \caption{نمودار کلاس طراحی سامانه‌ی کندو}
  \label{fig:1}
\end{figure}
\clearpage
این نمودار کلاس طراحی شامل 5 بسته اصلی می‌باشد که طبق معماری لایه‌ای سیستم این بسته‌ها انتخاب شده‌اند. این بسته‌ها عبارتند از: بسته رابط گرافیکی، بسته کنترلر، بسته سرویس، بسته مخزن، و بسته پایگاه داده. وظایف هر یک از این بسته ها در فصل 3 به طور کامل آورده شده است.

به طور کلی رابطه بین کلاس‌های رابط گرافیکی و کنترلر از نوع
\lr{call}
 است. همچنین رابطه بین کلاس‌های کنترلر و سرویس نیز از نوع
 \lr{call}
 می‌باشد و به همین شکل این رابطه بین کلاس‌های بسته سرویس و مخزن هم وجود دارد.
 همچنین رابطه بین کلاس‌های بسته سرویس و بسته مدل می‌تواند از هر ۳ نوع
 \lr{create}
 یا 
 \lr{use}
 یا 
 \lr{call}
 باشد. و درنهایت از آنجایی کلاس‌های بسته مخزن تنها از کلاس‌های بسته مدل در پارامترهای ورودی و خروجی متدهایشان استفاده می‌کنند رابطه بین آن‌ها از نوع 
 \lr{use}
 می‌باشد.
 
 برای جلوگیری از سربارگذاری 
 \lr{(Overloading)}
برای بسته‌های کنترلر، سرویس، و مخزن از کلاس‌های متعددی استفاده شده است و همه‌ی عملیات‌های آن‌ها در یک کلاس پیاده‌سازی نشده است. این کار علاوه‌بر جلوگیری از سربارگذاری، تغییر و استفاده مجدد از هر یک از این کلاس‌ها که به یک زیرسیستم خاص مربوط می‌شوند را ساده‌تر می‌کند.
\chapter{جمع‌بندی و تجارب کار گروهی}

\section{تجارب کار گروهی}
یکی از مهمترین دستاورهای هر یک از اعضای تیم، تجربه کار گروهی منظم در کنار هم و در تمام طول ترم بود که تا به اینجا همچین تجربه‌ای برای اعضا وجود نداشت. از آنجایی که شرایط خاص برگزاری ترم تحصیلی چالش‌های به ثمر رساندن یک کار تیمی را بیشتر می‌کرد و نیاز به وجود برنامه‌ای منظم برای به انجام رساندن این پروژه بیشتر احساس می‌شد. بنابراین تقریبا از زمان شروع پروژه به طور منظم هر هفته جلسات گروهی برگزار می‌شد تا هماهنگی میان اعضای تیم حفظ شود و کارها به‌موقع انجام شود.


از نکاتی که تیم توسعه بسیار برایشان جذاب واقع شد موضوع این پروژه بود که در رابطه با فرایند انتخاب واحد و ثبت‌نام بود. تیم توسعه به دلیل جایگاه خود که همگی دانشجو هستند به شکل ملموسی همواره با معضل انتخاب واحد برخورد داشته و دارند. روند کلی این پروژه و تصمیمات مختلفی که در طول فرایند این پروژه توسط تیم توسعه گرفته شد یک نتیجه ی بسیار جالب برای تیم توسعه داشت، آن هم این است که‌ فرایند انتخاب واحد تا حد زیادی یک فرایند بسیار گسترده می‌باشد که گروه‌های مختلفی از اشخاص را در برمی‌گیرد‌، اما پیاده‌سازی همین فرایند اگر به شکل پله پله صورت گیرد و برنامه ریزی اصولی برای آن انجام شود، و از همه مهم‌تر، تیمی که مسئول پیاده‌سازی سامانه‌ای برای این فرایند می‌باشند، شناخت کافی از دامنه‌ی کاربرد و دغدغه‌های دانشجویان و اساتید و معاونین داشته باشند تا چه حد می‌تواند به آسانی در مدت معینی پیاده‌سازی شود و در عین حال برای دوره‌های مختلف کارآمد باشد. جوری که دیگر نه تنها به انتخاب واحد به عنوان یک معضل نگاه نشود بلکه فرایندی باشد که کوچک‌ترین درگیری ذهنی و استرس را برای افراد مختلف که با آن در تعامل هستند به همراه داشته باشد. البته باید به این موضوع هم اشاره کرد که یکی از عوامل بسیار مهمی که به خوبی، تیم توسعه را در انجام پروژه یاری کرد آشنایی کاملی بود که از دامنه‌ی کاربرد برای اعضای تیم وجود داشت که این آشنایی نتیجه‌ی تجربیات شخصی اعضای تیم و همچنین مشورتی بود که با افراد مختلف صورت گرفته شده بود و لذا می‌توان گفت از مهم‌ترین نتایج این پروژه که قطعا در آینده برای تیم توسعه مفید می‌باشد این خواهد بود که در هر پروژه‌ی مهندسی نرم‌افزار تا چه حد شناخت دامنه‌ی کاربرد و تعامل با مشتری می‌تواند مفید واقع شود و باعث شود که حتی سنگین‌ترین پروژه‌ها نیز به یک پیاده‌سازی مطلوب ختم شوند.


همچنین باید گفت تیم توسعه بسیار علاقمند می‌باشد که موضوع این پروژه را تا مرز پیاده‌سازی هم پیش ببرد، چرا که در مرحله برنامه‌ریزی و طراحی‌ای که در این درس توسط تیم توسعه بر روی این موضوع صورت گرفت، جنبه‌های مختلف فرایند انتخاب واحد توسط تیم بررسی شدند، همچنین تعاملی که تیم توسعه بر روی این پروژه با همیار درس داشتند با توجه به آن که همیار درس نیز تسلط کافی بر روی دامنه ی کاربرد و جنبه‌های مختلف آن داشت از دیگر نکاتی می‌باشد که به شکل‌گیری بهتر این پروژه منجر خواهد شد و از همه مهم‌تر توصیه‌ها و راهکارهای استاد درس که در طول ترم و طراحی این پروژه به تیم توسعه، گوشزد می‌شد، همه چیز را آماده می‌کند که تیم توسعه را به این فکر بیاندازد که به سراغ پیاده‌سازی آن برود و تیم توسعه اطمینان دارد که این پروژه می‌تواند با این پیش زمینه‌ی کنونی و کامل خود به پیاده‌سازی بسیار خوبی نیز منجر شود، به طوری که بعد از پیاد‌ه‌سازی به نوبه‌ی خود بتواند بخش عمده‌ای از معضلات دانشگاه‌های کشور را که در ارتباط با فرایند ثبت‌نام و انتخاب واحد می‌باشد، برطرف سازد.
\section{تجربه‌های فردی}
علاوه‌بر تجربه کار گروهی که برای همه‌ی اعضای تیم مشترک بود هر یک از اعضا تجربه‌های فردی هم کسب کردند که در ادامه این تجربه‌ها از زبان هر یک از اعضا آورده شده است:

\begin{itemize}
\item
مقداد دهقان (سرگروه تیم):
''به شخصه به‌عنوان اولین تجربه سرگروهی یک تیم که هدف آن تحلیل و طراحی یک نرم‌افزار می‌باشد، می‌توانم بگویم که تا اینجا سنگین‌ترین کار تیمی خود را در دانشگاه داشته‌ام که با توجه به مجازی بودن ترم تحصیلی، سنگینی ابن کار به مراتب بیشتر از قبل بوده است. از تجربیاتی که سرگروهی یک تیم توسعه نرم‌افزار برای من داشته است می‌توانم به چالش های انتخاب مناسب‌ترین راه برای رسیدن به بهترین نتیجه بوده است.
مطمعنا اعمال قوانین چابکی در توسعه نرم‌افزار برای پیش‌برد بهتر پروژه کمک‌کننده خواهد بود. از دیگر نکاتی که در پیش‌برد بهتر پروژه کمک‌کننده بود تقسیم وظایف بر اساس توانایی‌های اعضا بود که باعث می‌شد سرعت انجام کارها بیشتر شود.‌‌``
\item
علی مراثی‌زاده:
''یکی از نکات جالب برای من شناخت مراحل مهمی بود که قبل از مرحله پیاده‌سازی وجود دارد. اینکه انتظار دریافت تمام ریزجزئیات از مشتری در همان ابتدای کار، انتظار بیجایی است. همچنین طراحی واسط کاربری قبل از اینکه اشراف خوبی بر روی نیازمندی‌ها و موردکاربرد‌ها وجود داشته باشد، خروجی مناسبی ندارد. مورد دیگر اینکه با توجه به فشاری که در طول انجام پروژه احساس شد و افزایش این فشار در روزهای پایانی و نزدیک به موعد تحویل، درک بهتری درباره لزوم اولویت‌بندی و تقسیم موردکاربردها به تکرار‌های مختلف در مراحل پیاده‌سازی پروژه احساس شد تا محصول را بتوان در موعد مقرر به مشتری تحویل داد بدون اینکه کیفیت موردکاربرد‌های مهم‌تر، از میزان مشخصی پایین‌تر برود. در کنار مباحثی که به‌طور مستقیم با درس ارتباط داشتند، طراحی رابط کاربری با \lr{AdobeXD}
 و راه‌اندازی سرور 
\lr{Redmine}
  نیز از تجربیات جدیدی بود که کسب کردم.``
\item
آرمین جعفرپیشه:
''در ابتدای هر چیز بابت داشتن هم‌تیمی‌های بسیار خوبم در این پروژه خدا را شاکرم و ابراز خرسندی می‌کنم و امیدوارم پروژه‌های دیگری را به همراه این تیم سپری کنم. اما در مورد پروژه‌ی درس تحلیل و طراحی باید بگم واقعا در طول این پروژه تجربیات زیادی کسب کردم که احتمالا در آینده‌، نه تنها در زمینه‌ی تحصیلی و شغلی، بلکه حتی در طول زندگی هم برایم مفید خواهد بود. از مهم‌ترین آن‌ها می‌تونم به کار گروهی مخصوصا در دوران مجازی، توجه به ددلاین تعیین شده در زمان‌های مختلف، نگارش متن به شکل اصولی و حس مسئولیت اشاره کنم. و اما در مورد محتوای پروژه هم قطعا تغییر رویکرد من نسبت به جایگاه مشتری و سعی در تعامل هر چه بیشتر با او و توجه ویژه به دامنه‌ی کاربرد و از همه مهم‌تر فهم این که پروژه‌های مهندسی نرم‌افزار شاید هیچ‌وقت به یک پاسخ مشخص ختم نشوند و همیشه برای بهبودشان راهی وجود دارد، و توازنی که باید بین این بهبودی و اتمام پروژه در نظر گرفته شود از مهم‌ترین دستاوردهای من در این پروژه می‌باشد که قطعا در آینده می‌تواند بسیار برایم مفید واقع شود.‍‍‍‍‍‍``
\item
محمد نصراصفهانی:
''تجربه پروژه گروهی، در فضای مجازی و به دور از هم آن هم به مدت چند ماه قطعا می‌تونه در حوزه کاری آینده مفید باشه و با توجه به همکاری همه اعضا، مسئولیت‌پذیری رو در همه ی ما تقویت کرد. باعث شد که کار با ابزارهای مختلف رو یاد بگیریم مثل ابزارهای کار مشارکتی و غیره مثلا ابزار طراحی گرافیکی. در کل تجربه خیلی خوبی بود و از گروهی که داشتم خیلی راضی بودم و اگر به عقب برگردم و یا در آینده پروژه‌ای داشته باشیم قطعا مشتاق همکاری باهاشون هستم.``
\item
مریم یزدی:
''به دست آوردن تجربه‌ای گران‌بها از کار گروهی به صورت دورکاری را بی‌شک می توان یکی از مهم‌ترین و ارزشمندترین نتایج این پروژه برای هر یک از اعضای تیم توسعه دانست. علاوه بر آن، این پروژه به ما (به‌عنوان یک دانشجو) کمک شایانی در درک نیازها و توانایی‌های لازم برای توسعه‌ی نرم‌افزار نمود. صرف تحلیل و طراحی نرم‌افزار و عدم تمرکز زیاد بر پیاده‌سازی  آن، دیدگاه جدیدی از این علم گسترده را برای ما به ارمغان آورد که شاید بدون مطالعه‌ی این درس هرگز به آن دست نمی‌یافتیم. اکنون با دید وسیع‌تری می توانیم به این‌گونه مسائل بنگریم و بدانیم که به دست آوردن یک نرم‌افزار خوب و کارآمد، نه تنها مستلزم پیاده‌سازی حرفه‌ای، بلکه تحلیل و برنامه‌ریزی اصولی و صحیح و رسیدن به درک درست از آنچه واقعا باید انجام گیرد است. هم‌چنین، اینکه موضوع پروژه طراحی سامانه‌ی برنامه‌ریزی درسی بود سبب شد تا ما از منظری متفاوت به سیستمی گسترده و پیچیده که اغلب اوقات با آن سر و کار داریم نگاه کنیم و بیشتر به جزئیات و نکات ریز آن (چه در قسمت تحلیل سامانه و روند کار آن و چه در قسمت پیاده‌سازی) توجه داشته باشیم.‍``
\item
ناصر فرج‌زاده:
''بزرگ‌ترین تجربه ای که من در این پروژه بدست آوردم رو می‌تونم کار گروهی عنوان کنم و اینکه یاد گرفتم توی یک کار گروهی ممکنه زمان‌هایی پیش بیاد که فشار کار روی هر یک از افراد زیاد باشه ولی باید صبر داشته باشیم و با همکاری بقیه اعضا کار رو جلو ببریم.
غیر از این توی این پروژه کار با ابزار 
\lr{Adobe XD}
 رو یاد گرفتم که برای زمینه کاری خودم هم در آینده می‌تواند مفید باشد.``

\end{itemize}



\section{نرم‌افزارهای استفاده شده}
برای تحلیل و طراحی کندو از ابزارها و نرم‌افزارهای زیر در موارد نیاز استفاده شده است:
\begin{itemize}
\item
\lr{:Microsoft Visio}
برای رسم نمودارهای
\lr{UML}
از نرم‌افزار
\lr{Visio}
استفاده شده است. در ابتدای شروع پروژه برای انتخاب ابزار مناسب ترسیم نمودارهای
\lr{UML}
چندین نرم‌افزار دیگر همچون
\lr{enterprise architect}
و 
\lr{visual paradigm}
بررسی شدند و درنهایت از بین این ۳ ابزار، مناسب‌ترین آن‌ها با نظر اعضای تیم انتخاب شد.

\item
\lr{:Latex}
برای نگارش متن سند پروژه از
\lr{Latex}
استفاده شده است و تا حد امکان سعی شده است که قواعد نگارشی رعایت شود و متن نهایی اشتباهات کمی داشته باشد.
\item
\lr{:Redmine}
در ابتدا برای مدیریت پروژه از ابزار
\lr{trello}
استفاده شد اما برای برنامه ریزی بهتر و استفاده از نمودارهای
\lr{gantt}
که روند پیشرفت پروژه را بهتر نشان می‌دهند، تصمیم گرفته شد که از نرم‌افزار
‌\lr{Redmine}
استفاده شود.
\item
\lr{:Adobe XD}
برای طراحی رابط گرافیکی اولیه که مطابق با جداول موردکاربردهای گسترده بودند از نرم‌افزار
\lr{Adobe XD}
استفاده شده است.
\item
\lr{BBB}
\LTRfootnote{Big Blue Button}
:
برای برگزاری جلسات هفتگی از سامانه شاتل استفاده شد که بر پایه پلتفرم
\lr{BBB}
می‌باشد.
\item
\lr{:Google sheets}
از آنجایی که همه‌ی اعضای تیم از سیستم‌عامل ویندوز و نرم‌افزار
\lr{Microsoft Word}
استفاده نمی‌کردند برای نگارش متن‌های اولیه سند که به هر نفر واگذار می‌شد از ابزار
\lr{google sheets}
استفاده شد تا همه‌ی افراد قادر به استفاده از فایل‌های سایر اعضای تیم باشند.
\end{itemize}


\end{document}
